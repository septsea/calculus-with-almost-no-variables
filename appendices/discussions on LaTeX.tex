\chapter{\theauthor{}论\texorpdfstring{\gls{latex}}{乳胶}}

\begin{remark*}
    本章是用来测试排版的.
\end{remark*}

我将在本章讨论跟\gls{latex}有关的东西.

\section{一些说明}

我建议您安装\gls{latex};
这样, 您就可以跟着我 (或者, 本书) 操作了.

怎么装\gls{latex}呢?
您可以去看一看王然的%
\href{http://mirrors.ctan.org/info/install-latex-guide-zh-cn/install-latex-guide-zh-cn.pdf}%
{《一份简短的关于 \hologo{LaTeX} 安装的介绍》};
我认为这份安装说明还是挺好理解的.

我现在假定您\emph{安装了\gls{latex}},
并且您\emph{能用\gls{latex}写简单的小文档};
同时, 我还假定您最近更新了\gls{latex}的包
(假如您不知道怎么更新\gls{latex}的包,
请参考上面提到的\gls{latex}安装指南的 ``升级宏包'' 的部分).

我提一提路径的事项.
您现在应该在阅一个叫 \verb`main.pdf` 的文件, 对罢?
我称这个文件所在的目录为 \verb`./`~.
那么, \verb`./main.pdf` 就是您正在读的文件.
类似地, \verb`./main.tex` 就是 \verb`./main.pdf` 的 ``父亲'';
具体地, \gls{latex}可编 \verb`./main.tex`
(与别的一些文件) 为 \verb`./main.pdf`~.
在 \verb`./` 下, 您可以看到一些别的目录;
别的目录下也会有一些文件.
比如, \nameref{lst:sparrow} 就是一只 ``\gls{latex}麻雀''
(我在后面还会提到此事).

本书涉及的所有文字文件的编码都是\emph{\gls{utf8}};
同时, 我使用如下的指令编译\gls{latex}代码:
\begin{lstlisting}[
    language=bash,
    morecomment={[s]{<}{>}},
    caption={本书用到的\gls{latex}编译设定},
    label={lst:theWayOfCompilingTexFiles},
    frame=single
]
latexmk -xelatex -shell-escape -file-line-error -norc -synctex=1 -interaction=nonstopmode <file>
\end{lstlisting}
其中 \lstinline[language=bash,morecomment={[s]{<}{>}}]`<file>`
就是待编译的\gls{latex}文件的名字 (不含后缀 \verb`.tex`);
具体一些, 是指含 \verb`\begin{document}`
与 \verb`\end{document}` 的文件.
比如, 为编译此书, 可执行
\begin{lstlisting}[language=bash,morecomment={[s]{<}{>}}]
latexmk -xelatex -shell-escape -file-line-error -norc -synctex=1 -interaction=nonstopmode ./main
\end{lstlisting}
(当然, 您可能需要适当地调整 \verb`./`~,
使其含一个名为 \verb`main.tex` 的文件.)

一般地, 我不会详细地讨论一个\gls{latex}包;
所以, 欲了解更多的信息, 可在终端里执行
\begin{lstlisting}[
    language=bash,
    morecomment={[s]{<}{>}},
    caption={阅\gls{latex}包的文档},
    label={lst:readingTheDocOfAPackage},
    frame=single
]
texdoc <name_of_the_package>
\end{lstlisting}
以阅包的文档;
其中
\lstinline[language=bash,morecomment={[s]{<}{>}}]`<name_of_the_package>`
就是包的名字.
例如,
若我想了解%
可自由定制页眉跟页脚的包 \textsf{fancyhdr} 的信息,
我可以在终端执行
\begin{lstlisting}[language=bash,morecomment={[s]{<}{>}}]
texdoc fancyhdr
\end{lstlisting}
稍等片刻, 您就可以看到一个名为 \verb`fancyhdr.pdf` 的文档.

值得一提的是,
大部分包的文档都是用\gls{eng}写的 (如 \textsf{fancyhdr});
所以, 您最好具备一定的读\gls{eng}的能力.
当然了, 也有不少%
\href{https://ctan.org/topic/chinese-doc}{\emph{\gls{chi}的资料}}.

本章提及的代码要么在
\verb`./appendices/discussions on LaTeX.tex` 里 (较短的代码),
要么在 \verb`./listings/` 目录下的文件里 (较长的代码).
使用任何一个文字编辑器即可看到代码.
建议您善用文字编辑器为您提供的查找功能.
一般地, 按 \verb`Ctrl` 键不放
(假如您使用水果系统, 请按 \verb`Command` 键),
再按 \verb`f` 键, 最后释放这二个键,
应该就可以搜索文件内的文字了.
假如您用维姆或依麦克斯文字编辑器,
想必您比我更懂如何查找文字, 对罢?

一般地, 我推荐使用\gls{vsc}写\gls{latex}文件.
\glssymbol{vsc}是一个现代的、美观的、强大的文字编辑器;
您可以在互联网上找到关于它的更多信息.

我假定您安装了这款编辑器, 并打算用它写\gls{latex}.
那么, 按 \verb`F1` 键, 删去大于号 \verb`>`~, 输入
\begin{lstlisting}
ext install james-yu.latex-workshop
\end{lstlisting}
再按 \verb`Enter` 键,
即可安装\glssymbol{vsc}的\gls{latex}插件 LaTeX Workshop.
然后, 按 \verb`F1` 键, 输入 \verb`open settings json`~,
触发 ``Preferences: Open Settings (UI)'',
就可以看到一个名为 \verb`settings.json` 的文字文件.
在最外层的 \verb`{` 与 \verb`}` 里, 追加
\begin{lstlisting}[
    frame=single,
    caption={为\glssymbol{vsc}准备的\gls{latex}编译设定},
    label={lst:aLatexRecipeForVsc},
]
"latex-workshop.latex.recipe.default": "latexmk",
"latex-workshop.latex.recipes": [
    {
        "name": "latexmk",
        "tools": [
            "latexmk"
        ]
    }
],
"latex-workshop.latex.tools": [
    {
        "args": [
            "-xelatex",
            "-shell-escape",
            "-file-line-error",
            "-norc",
            "-synctex=1",
            "-interaction=nonstopmode",
            "%DOCFILE%"
        ],
        "command": "latexmk",
        "name": "latexmk"
    }
],
\end{lstlisting}
假如您看到 ``Expected comma'' 的错误, 那么, 在
\begin{lstlisting}
"latex-workshop.latex.recipe.default": "latexmk",
\end{lstlisting}
的上一行的末尾加一个 \verb`,` (逗号).
最后, 记得保存文件.
这样, 您就不必每次都输入较长的编译命令了.

您可以自行在互联网上了解 LaTeX Workshop 的用法;
我就不继续展开了.
(当然, 假如您愿意, 此插件自带的\gls{eng}文档就很不错.)

\section{麻雀}

或许, 您知道, 麻雀是一种小但完整的生物.
That is to say,
a sparrow is something small but complete in every detail.
在学习新的东西时, 麻雀是一种对初学者相当友好的东西;
我喜欢, 您应该也不讨厌.

下面就是一只\gls{latex}麻雀 (\gls{eng}: \hologo{LaTeX} sparrow).
具体地, 下面的代码虽小,
但充分体现了用\gls{latex}写\emph{\gls{chi}}的\emph{书}时的基本要素.

\lstinputlisting[
    frame=single,
    language=TeX,
    caption={\lstinline`./listings/sparrow.tex`},
    label={lst:sparrow},
    showspaces=true
]{listings/sparrow.tex}

既然您可以正常地用\gls{latex}, 那就试编译这只\gls{latex}麻雀罢.
您也不必手动复制本章的代码;
我早已准备了一个独立的\gls{latex}文件供您编译
(首行代码的上面的那行字告诉您这只麻雀的家在哪儿).

您应该能编出一个 23~页的\emph{\gls{pdf}}文件, 对罢?
如果没有问题, 我就继续讲解了.

一般地, 一个完整的\gls{latex}文件由二个部分作成:
导言区 (\gls{eng}: preamble) 与文档区.
所谓导言区, 就是 \verb`\begin{document}` 前面的内容.
这么看来, 这只\gls{latex}麻雀的导言区就是
\begin{lstlisting}[language=TeX]
\documentclass[fontset=fandol]{ctexbook}

\title{这是书的标题}
\author{这是作者}
\date{这是日期}
\end{lstlisting}

可以看到, \verb`\documentclass` 后面接了一些东西.
\verb`{ctexbook}` 告诉\gls{latex},
这个文件使用名为 \textsf{ctexbook} 的文档类 (\gls{chi}的书);
\verb`[fontset=fandol]` 是所谓的 ``可选参数'' (按需求加).
具体地, \verb`fontset=fandol` 指使用%
名为 fandol 的一套 (用于\gls{chi}的) 字体.
原则上, 我可以不加这个可选参数;
不过, 我还是加上了它.
为什么呢?
其实这得问问\gls{marcohard}公司
为什么\emph{\gls{eng}}的\gls{windows}不带黑体、楷体与仿宋罢.
假如您使用\emph{\gls{chi}}的\gls{windows},
那么您可以删去麻雀的 \verb`[fontset=fandol]`~,
再试着编译,
以证明这确实只是一个 ``可选参数''.

后面的
\verb`\title`~, \verb`\author` 跟 \verb`\date`
的作用就很明显了.
就算您看不懂长长的\gls{eng}文档, 您也一定能看出,
这分别是 ``标题'' ``作者'' ``日期'' 的意思.
使用方法也很简单:
在花括号里输入对应的信息即可.
值得一提, 假如您不希望指定日期, 您可以留白;
也就是说, 写 \verb`\date{}` 即可.
假如您希望指定今日的日期,
您可以向花括号内写入 \verb`\today`;
也就是说, 写 \verb`\date{\today}` 即可.
那么, 您今日编它是一个结果, 而明日编它就又是另一个结果.

这只麻雀的导言区倒是没什么亮点;
所以, 现在看一看文档区.
文档区就是
\verb`\begin{document}` 到 \verb`\end{document}` 的内容.
具体地, 就是码~\ref{lst:sparrow} 去除其导言区后的内容.

我们深入文档区罢.

我先说一下 \verb`%` 的作用.
\emph{一般地}, 假如\gls{latex}代码的一行里出现了 \verb`%`~,
那么从 \verb`%` 到该行的末尾的所有字符都会被忽视.
所以,
\lstinline[language=TeX]`% 正式开始文档.`
就是不会被\gls{latex}所处理的内容.
我们可利用此特性, 为\gls{latex}代码插入注释.
当然, 假如我们需要在文档里输入百分号, 可输入 \verb`\%`~;
这里, 反斜线 \verb`\` 起到了 ``转义'' 的作用:
原本\gls{latex}会忽视一行内 \verb`%` 到行尾的字符,
但 \verb`\` 却 ``改变'' 了 \verb`%` 的含义, 使百分号被输出.

一般地, 一本书有三个部分:
正文前的资料 (\gls{eng}: front matter),
正文的主要部分 (\gls{eng}: main matter),
正文后的附加资料 (\gls{eng}: back matter).
正文前的资料一般包括封面、目录、前言等内容.
正文的主要部分一般就是书的主体.
正文后的附加资料一般包括参考文献、索引、后记等内容.
跟正文的主要部分相比,
正文前的资料与正文后的附加资料一般都不带章的编号
(直白地, 书有第一章, 第二章等章节,
但一般都不说 ``目录是第几章'' ``后记是第几章'', 对罢?).
并且, 经验告诉我们, 正文前的资料的页码一般不用~1, 2, 3,
而是用罗马数字~i, ii, iii.
然后, 正文的主要部分的页码一般会被重置为~1, 并使用阿拉伯数字.
最后, 正文后的附加资料的页码一般不会被重置,
而是延续正文的主要部分的页码的风格.

现在您应该不难理解 \verb`\frontmatter` 的作用了.
它就是告诉\gls{latex}, 如果不遇见 \verb`\mainmatter`~,
那么不给章编号, 且页码使用罗马数字.
具体地, 在这只麻雀里, 下面的内容都是 ``正文前的资料'':

\begin{lstlisting}[language=TeX]
\maketitle% 利用导言区的信息作一个简洁的封面.

\tableofcontents% 作一个目录.

\listoftables% 作一个展示书的表格的位置的目录.

\chapter{前言}% 插入新的一章. 注意, 本章不被编号.

% 下面的都是前言里的内容.
我想向乳胶问好.
\end{lstlisting}

您可能会注意到, 跟码~\ref{lst:sparrow} 相比,
我添加了一些以 \verb`%` 开头的文字:
这些都相当于是注释, 解释了命令的功能.
(所以, 原则上,
我完全可在码~\ref{lst:sparrow} 里加入这些东西;
不过, 为了简洁, 我还是选择单独提取相应的代码加注释.)

有一个小细节值得一提: ``目录'' 跟 ``前言'' 前都有空白页.
这是因为, \gls{latex}默认会使章从奇数页 (右手边) 开始.
当然, 有简单的办法取消这个设定;
不过, 我暂时不提此事.
具体地, 这只麻雀的第~ii, iv, vi, viii~页都是空的.

好.
现在我们进入 \verb`\mainmatter`~.
从这儿到 \verb`\backmatter` 前的章都会被编号;
并且, 页码被重置, 且使用阿拉伯数字编号.
具体地, 这只麻雀的第~viii~页的下一页就是第~1~页.

又有一个小细节值得一提:
虽然这只麻雀的 ``第一部分\quad 甲'' 下方标注的页码为~1,
但您的\gls{pdf}文件阅读器会认为它在第~9~页.
同时, 这只麻雀的封面并不在第~i~页, 而是在第~1~页.
作为对比, 您现在正在看的 \verb`main.pdf` 就没有这个\gls{bug};
本书的封面在第~i~页.
此\gls{bug}可被一个名为 \textsf{bookmark} 的包修复.
具体地, 在导言区里加入一行字
\begin{lstlisting}[language=TeX]
\usepackage{bookmark}
\end{lstlisting}
保存文件, 然后再试着编译~\nameref{lst:sparrow}.
那么, 这只麻雀所显示的页码就会与%
\gls{pdf}文件阅读器的页码保持一致.
顺便, 您还会发现, 这只麻雀还多了一些 ``书签'';
您用\gls{mouse}点一点,
阅读器就会跳到您刚点的地方.

我顺便提一提 \verb`\usepackage` 罢.
正如其字面意思那样,
这句话可导入被 \verb`{` 与 \verb`}` 所包围的那个包.
当然, 一般地, 就像文档类有可选参数那样, 包也有可选参数.
在 \verb`\usepackage` 后加入一对方括号,
然后向方括号内填入需要的东西,
就可以实现自己想要的效果.
可向方括号内写入的内容一般都会被包的文档提及
(您还记得怎么看一个包的文档吗?).

我们看 ``第一章\quad 我向乳胶问好''.
您可能会注意到, ``您好, 乳胶'' 的右上角有一个小小的数字~1;
这其实是所谓的脚注编号. 往脚 (页底) 看, 您就会看到一条注解,
``解释了'' \gls{latex}为何物.
命令 \verb`\footnote` 即可实现此事;
在需要脚注的地方后写 \verb`\footnote`~,
再写一对花括号, 再往花括号里填脚注即可.
您可能会问:
``1'' 在哪儿?
为什么不需要指定编号呢?
一般地, \gls{latex}会自动地计算编号;
您一般也不愿手动指定 1, 2, 3 罢.

您可能经常看到像 ``\LaTeX{}'' 这样交错的五字母;
这算是\gls{latex}的 ``彩蛋'' 罢.
事实上, 单独的 \verb`\LaTeX` 就可输出这样的文字.
那么, 为什么还要一对花括号呢?
那是因为有些人打字时, \gls{eng}跟\gls{chi}间是不带空格的:
\begin{lstlisting}[language=TeX]
\documentclass[fontset=fandol]{ctexbook}

\begin{document}
打游戏还是得用Windows系统
\end{document}
\end{lstlisting}
您可试编译上面的\gls{latex}代码;
因为它比较简短, 我就不单独提供文件了.
不出意外, 您可以看到 ``打游戏还是得用 Windows 系统''.
在代码里, 我没加空格;
不过, 它的输出里, ``Windows'' 的前后都有空白.

那么, 假如他想输出 ``我觉得 \LaTeX{} 真不错'', 会怎么打?
一个可能的版本如下:
\begin{lstlisting}[language=TeX]
\documentclass[fontset=fandol]{ctexbook}

\begin{document}
我觉得\LaTeX真不错
\end{document}
\end{lstlisting}
您可试编译上面的\gls{latex}代码;
因为它比较简短, 我就不单独提供文件了.
不出意外, 您会 ``逝世''.
事实上, 这么打代码, \gls{latex}会认为
\verb`\LaTeX真不错`
是一个命令;
可惜, 这是一个未被定义的东西,
故\gls{latex}无法编译上面的\gls{latex}代码.

对此, 他有三种解决办法. 他可以加空格:
\begin{lstlisting}[language=TeX]
我觉得 \LaTeX 真不错
\end{lstlisting}
当然, 假如他不想加太多空格, 则可考虑
\begin{lstlisting}[language=TeX]
我觉得\LaTeX 真不错
\end{lstlisting}
假如他就是不想加空格, 那就得这样了:
\begin{lstlisting}[language=TeX]
我觉得\LaTeX{}真不错
\end{lstlisting}
这里, \verb`{}` 就隔开了命令 \verb`\LaTeX`
跟后面的文字 \verb`真不错`~.

值得注意的是, 这只麻雀的第~1~章的第~1~段是
\begin{quotation}
    您好, 乳胶\textsuperscript{1}!
    或许, 我应该叫您的原名, \LaTeX{}.
\end{quotation}
而不是
\begin{quotation}
    您好, 乳胶\textsuperscript{1}!
\end{quotation}
为什么呢?

您用过\gls{word}或 WPS~文字罢?
当您在一段里输入较多文字时,
您会发现, \gls{word}或 WPS~文字自动地断行.
毕竟, 这是所谓的 ``所见即所得'' 的文字处理应用.
不过, 我们一般都是用各种各样的文字编辑器写\gls{latex}.
所以, 就会有这样的情况:
有的文字编辑器不支持自动地断行
(也就是说, 就算一行文字很长,
超过了屏幕的显示范围, 它还是不会断此行);
有的文字编辑器支持自动地断行,
不过, 因为各种各样的原因, 这个功能并没有被启用.
至少, 我很清楚,
不少主流的代码编辑器都不会自动地断行,
除非您自己更改这个设定.
不过, 据我所知, 初始设定的力量还是很强大的:
很多人不用 ``更好用的设定'',
仅仅是因为初始的设定不是 ``更好用的设定'' 罢了.
而且, ``更好用的'' 往往因人而异.

\gls{latex}就 ``照顾'' 那些不自动地断行的文字编辑器.
具体地, \emph{简单地 ``换行'' 只是加空格,
    而 ``二次换行'' 才是另起一段.}
也就是说, 一般地, 在\gls{latex}里, 我们用空行隔开二段.
所以,
\begin{quotation}
    I wantto learn 100\% about \LaTeX{}.
\end{quotation}
是第~2~段.

或许您会好奇, 为什么这只麻雀给出 ``wantto''.
这也很简单: 我在 \verb`want` 后加了 \verb`%`~.
您还记得 \verb`%` 的作用罢?
一般地, 假如\gls{latex}代码的一行里出现了 \verb`%`~,
那么从 \verb`%` 到该行的末尾的所有字符都会被忽视.
这么看来, 这里的 \verb`%` 只是忽视了自身才对呀?
您还记得, ``一次换行'' 只是加空格罢?
\verb`%` 其实也忽视了这个 ``相当于空格的换行''.
并且, \gls{latex}\emph{一般}会忽视行首的空白.
所以, 就算 \verb`to` 前有 4~个空格, 它还是无作用.
所以, 这只麻雀给出 ``wantto''.
这也顺便解释了为什么 \verb`100\%` 前的 8~个空格被忽略.

可用 \verb`\section` 插入新的一节.
它的用法跟 \verb`\chapter` 的类似, 所以我不细说了.

接下来, 您可以看到一些\gls{math}公式.
我不在这里教您输入\emph{具体的}\gls{math}公式;
也就是说,
\emph{我不教您\gls{math}符号与\gls{latex}指令的对应关系.}
您可以参考别的资料学习它.
我简单说几句话.
一般地, \gls{math}公式有 ``显示型'', 还有 ``行内型''.
``显示型'' 一般是独立成行的公式的样式;
``行内型'' 一般是被嵌入于内文的公式的样式.

一般地, 欲输入行内公式,
用二个 \verb`$` 包围公式即可.
不被二个 \verb`$` 包围,
\verb`ABC` 就只是普通的一个\gls{eng}词 ABC;
被二个 \verb`$` 包围,
\verb`$ABC$` 就是\gls{math}公式 $ABC$.

一般地, 欲输入显示公式, 有二个常用的方法.
一个是不带编号的 \verb`\[` 与 \verb`\]`~;
一个是带编号的 \verb`equation` 环境.
假如您希望您待输入的公式无编号,
用 \verb`\[` 与 \verb`\]` 包围它;
假如您希望您待输入的公式有编号,
用 \verb`\begin{equation}` 与 \verb`\end{equation}` 包围它.

一般地, 您还可方便地引用带编号的公式.
实现此事需二步: 加标签, 再引用它.
在 \verb`\begin{equation}` 后, 加入 \verb`\label`~,
再写一对花括号, 再往花括号里填一个 (唯一的) 标签.
如:
\begin{lstlisting}[language=TeX]
\begin{equation}\label{eq:Euler}
    \mathrm{e}^{\mathrm{i} x} = \cos {x} + \mathrm{i} \sin {x}.
\end{equation}
\end{lstlisting}
我使用了 \verb`eq:Euler`~.
我推荐您像我一样为公式加标签.
\verb`eq` 告诉我, 这是公式 (\gls{eng}: equation);
\verb`Euler` 告诉我, 这是跟算学家 Euler 有关的公式.
于是, 我可如此引用此公式:
\begin{lstlisting}[language=TeX]
据说, 算学家 Leonhard Euler 早就发现式~(\ref{eq:Euler}).
今日, 大家一般都叫它 Euler 公式.
\end{lstlisting}
这二行字给出
\begin{quotation}
    据说, 算学家 Leonhard Euler 早就发现式~(1.1).
    今日, 大家一般都叫它 Euler 公式.
\end{quotation}
可以看到, 一般地,
\verb`\ref{eq:Euler}` 给出 ``纯的编号'':
不带 ``式'', 也不带括号.

我顺便解释一下 \verb`~` 是什么罢.
它其实就是一个空格.
不过, 它是一个 ``不可被断行的'' 空格.
在什么时候有用呢?
譬如, ``式~(1.1)'' 的 ``(1.1)'' 不宜出现于行首, 是罢?
``Mrs.~Mori'' 也不宜跨越二行, 对罢?
这个时候, 可用 \verb`~`~;
它告诉\gls{latex}, 不要在此处断行.

我们接着看这只麻雀.
不出意外, 您应该会在下一页看到二条脚注.
不出意外, 二条脚注的编号分别为 2 与 3.
一般地, \gls{latex}按章为脚注编号, 而不是按页.

您会发现, 若在一行后插入 \verb`\\`~,
则\gls{latex}强行在此处断行 (但不分段);
若在一行后插入 \verb`\par`~,
则\gls{latex}强行在此处分段 (自然地, 也断行了).
我曾说,
\emph{一般地}, 一次换行是加空格, 而二次换行是分段.
现在, 您也知道了, 也有强制断行与分段的方法.
不过, 一般地, 不要用它们太多次.

类似地, 您可用 \verb`\clearpage` 强行换页.

不过, 有意思地, 为什么
\begin{lstlisting}[language=TeX]
可用 \verb/\clearpage/ 手动换页.
\end{lstlisting}
里的 \verb`\clearpage` 不换页呢?
这是一个好问题; 好问题值得有好解答.
很简单: 可用 \verb`\verb` 输出一些需
``原样打印'' (\gls{eng}: to be printed verbatim) 的文字.
您可能会问: 假如待原样打印的文字带 \verb`/` 怎么办?
很简单; 换别的.
\verb`\verb:g/f:` 就是 \verb:g/f: .
当然了, \verb`\verb+g/f+` 也是 \verb+g/f+ ;
类似地, \verb`\verb?g/f?` 还是 \verb?g/f? .
一般地, 待原样打印的文字里, 总有没被用到的符号罢?
挑一个没被用到的符号 $s$,
再使待原样打印的文字被二个 $s$ 包围即可.

我也提一提\gls{latex}的书的页眉与页脚的初始设定.
一般地, 章从右边 (奇数页) 开始.
章的首页的页眉无内容, 页脚有被居中放置的页码.
章的偶数页的页脚无内容, 页眉为页码跟章名.
章的奇数页 (除首页) 的页脚无内容, 页眉为节名跟页码.
章名与节名都被设为斜体 (不过, 土话除外).
章名与节名都被全部大写.

我们看完了 ``第一章''.
接着, 我们看 ``第二章\quad 不知道写什么了''.

可以看到,
``我问不下去了啊'' 的右上角有一个小小的 1.
没错, 这是脚注;
也正如我所言, 一般地, 脚注按章编号.

您还能看到一个表格.
一般地, 在\gls{latex}里, 欲作一个表格,
用 \verb`tabular` 环境.
譬如, 用
\begin{lstlisting}[language=TeX]
\begin{tabular}{l c r}
    胡话   & English       & 1   \\
    土话   & Chinese       & 22  \\
    乳胶   & \LaTeX{}      & 333 \\
    算学   & mathematics   & 4   \\
    算学家 & mathematician & 05  \\
\end{tabular}
\end{lstlisting}
可作出如下表格:
\begin{quotation}
    \begin{tabular}{l c r}
        胡话  & English       & 1   \\
        土话  & Chinese       & 22  \\
        乳胶  & \LaTeX{}      & 333 \\
        算学  & mathematics   & 4   \\
        算学家 & mathematician & 05  \\
    \end{tabular}
\end{quotation}

一般地, 用 \verb`&` 分隔表格的列,
用 \verb`\\` 分隔表格的行
(当然, 您可能还记得 \verb`\\` 的其他的用途).

可以看到, 这个表格无任何边框,
且第~1, 2, 3~列分别左、中、右对齐.
\verb`l`~, \verb`c`~, \verb`r`
分别对应\gls{eng}的
left, center, right.

假如我们要为表格绘外边框, 可用
\begin{lstlisting}[language=TeX]
\begin{tabular}{|l c r|}
    \hline
    胡话   & English       & 1   \\
    土话   & Chinese       & 22  \\
    乳胶   & \LaTeX{}      & 333 \\
    算学   & mathematics   & 4   \\
    算学家 & mathematician & 05  \\
    \hline
\end{tabular}
\end{lstlisting}
这可作出
\begin{quotation}
    \begin{tabular}{|l c r|}
        \hline
        胡话  & English       & 1   \\
        土话  & Chinese       & 22  \\
        乳胶  & \LaTeX{}      & 333 \\
        算学  & mathematics   & 4   \\
        算学家 & mathematician & 05  \\
        \hline
    \end{tabular}
\end{quotation}

当然, 我们还可作出稍复杂的表格:
\begin{lstlisting}[language=TeX]
\begin{tabular}{||l c|r||}
    \hline\hline
    胡话   & English       & 1   \\
    土话   & Chinese       & 22  \\
    乳胶   & \LaTeX{}      & 333 \\
    算学   & mathematics   & 4   \\
    \hline
    算学家 & mathematician & 05  \\
    \hline\hline
\end{tabular}
\end{lstlisting}
这可作出
\begin{quotation}
    \begin{tabular}{||l c|r||}
        \hline\hline
        胡话  & English       & 1   \\
        土话  & Chinese       & 22  \\
        乳胶  & \LaTeX{}      & 333 \\
        算学  & mathematics   & 4   \\
        \hline
        算学家 & mathematician & 05  \\
        \hline\hline
    \end{tabular}
\end{quotation}

总之, 在表格的行的相应位置插入 \verb`\hline`,
可绘制水平的线;
而在紧跟 \verb`\begin{tabular}` 后的
\verb`{` 跟 \verb`}` 里,
用 \verb`|` 可绘制竖直的线;
空格代表此处不绘制竖直的线.

我就不继续讲制表的细节了;
这不是我的主要任务.
您可以考虑阅专门的\gls{latex}教程.

您还看到了 ``表~2.1: 术语表'' 罢?
我讲讲如何作出这样的东西.

事实上, \verb`tabular` 是一个相对简单的表环境.
它无法处理稍复杂的需求.
所以, 我们要组合多个环境.
可用 \verb`table` 环境为表加题注与 (自动) 编号:
\begin{lstlisting}[language=TeX]
\begin{table}[h!]
    \caption{术语表}
    \begin{tabular}{l c|r}
        \hline
        胡话   & English       & 1  \\
        算学家 & mathematician & 05 \\
        \hline
    \end{tabular}
\end{table}
\end{lstlisting}
这可作出
\begin{table}[h!]
    \centering
    \caption{术语表}
    \begin{tabular}{l c|r}
        \hline
        胡话  & English       & 1  \\
        算学家 & mathematician & 05 \\
        \hline
    \end{tabular}
\end{table}

这里, 紧跟 \verb`\begin{table}` 后的 \verb`[h!]` 告诉\gls{latex},
若空间允许, 则在 ``此处'' (\gls{eng}: here) 插入表格.
假如您不加 \verb`[h!]`~, 那么它可能会跑到页的上方.
\verb`\centering` 要求\gls{latex}居中表格.
用 \verb`\caption` 即可为表添加题注.
因为我置它于 \verb`tabular` 前, 故题注在表的上方.

那么, 又要如何提及表格呢?
您可能还记得,
可用 \verb`\label` 为公式加标注,
并用 \verb`\ref` 提及公式.
我们也可施此法于表格.
具体地, 在
\lstinline[%
    language=TeX,morecomment={[s]{<}{>}}%
]`\caption{<您的题注>}`
后加 \verb`\label`~,
再写一对花括号, 再往花括号里填一个 (唯一的) 标签.
引用时, 用 \verb`\ref`~.

这是一个具体的例:
\begin{lstlisting}[language=TeX]
我在表~\ref{tab:terms} 列出了一些被我用,
但在其他场合不常被见到的术语.

\begin{table}[h!]
    \centering
    \begin{tabular}{l c|r}
        \hline
        胡话   & English       & 1  \\
        算学家 & mathematician & 05 \\
        \hline
    \end{tabular}
    \caption{术语表}\label{tab:terms}
\end{table}
\end{lstlisting}
这可作出
\begin{quotation}
    我在表~\ref{tab:terms} 列出了一些被我用,
    但在其他场合不常被见到的术语.

    \begin{table}[h!]
        \centering
        \begin{tabular}{l c|r}
            \hline
            胡话  & English       & 1  \\
            算学家 & mathematician & 05 \\
            \hline
        \end{tabular}
        \caption{术语表}\label{tab:terms}
    \end{table}
\end{quotation}

我就说这么多作表的事罢.

接下来, 就是 ``第二部分\quad 乙''.
其实这没什么亮点:
我也只是展现关于
\verb`\part`~, \verb`\chapter`~,
\verb`\section`~, \verb`\subsection`~,
\verb`\subsubsection`~,
\verb`\paragraph`~, \verb`\subparagraph`
的用法而已.

接下来, 您可以看到 ``附录~A\quad 乳胶的历史''.
这是 \verb`\appendix` 干的好事.
原本, 这应当是 ``第四章\quad 乳胶的历史'';
不过, 我加了魔法, 故从此之后 (到 \verb`\backmatter` 前)
章被编号为 A, B, C 等大写字母.
相应地, 节就是 A.1, A.2, A.3 等.
不过, 页码并没有被重置.

最后, 您可以看到 ``后记'';
因为我加了 \verb`\backmatter`~, 故此章不被编号.

或许, 我该指出一个事实:
\verb`\frontmatter`~, \verb`\mainmatter`~,
\verb`\appendix`~, \verb`\backmatter`
都不是必要的.
毕竟, 不是所有的书都有附录;
不是所有的书都会在最后一个带编号的章后加不带编号的章;
也不是所有的书都要目录 (尤其是小篇幅的作品).
我在前面举的一些例里没说这四句话.

好了.
我觉得我已经讲完了这只麻雀.
我希望本节能帮您更好地理解\gls{latex}.

\section{各种各样的东西}

接下来, 我简单地提及我怎么作出本书罢.

我使用 \textsf{ctexbook} 写书.
事实上, 这并不是一个特别好用的文档类.
据我所知, 专业的有 \textsf{memoir} 跟 \hologo{KOMAScript}.
不过, 它们对\gls{chi}的支持没有 \textsf{ctexbook} 的那么好.
这是一本\gls{chi}书,
所以, 我还是决定,
充分利用自己的蹩脚的\gls{eng}%
阅包的文档%
与
\href{https://tex.stackexchange.com/questions/}{%
    \hologo{LaTeX} Stack Exchange%
}.
毕竟, \gls{latex}是胡人的东西;
而且, 跟\gls{word}或 WPS~文字相比,
\gls{latex}确实不那么好上手.
而且, 要想用好\gls{latex}, 最好自己有一定的动手能力.

在下面的讨论里,
我假定您按码~\ref{lst:theWayOfCompilingTexFiles}
或码~\ref{lst:aLatexRecipeForVsc}
编译\gls{latex}文件;
我假定您的目标是写一本以\gls{chi}为主的书;
我假定您也用 \textsf{ctexbook}.

\subsection{修改字体}

本书选用了一些跟初始字体不一样的字体.

简单地, 可用 \verb`\setmainfont` 修改\gls{eng}的字体,
用 \verb`\setCJKmainfont` 修改\gls{chi}的字体;
不过, 具体地,
因为 \textsf{ctexbook} 有 ``默认\gls{chi}字体'',
故我们要先取消它们.
此事不难:
指定 \textsf{ctexbook} 的一个选项即可:
\begin{lstlisting}[language=TeX]
\documentclass[fontset=none]{ctexbook}
\end{lstlisting}
当然, 若您指定了其他的选项, 您自行添上相关设定即可.
譬如, 我假定您已使用如下的设定:
\begin{lstlisting}[language=TeX]
\documentclass[a4paper,UTF8]{ctexbook}
\end{lstlisting}
那么, 您在 \verb`[` 后加入 \verb`fontset=none,`~, 即
\begin{lstlisting}[language=TeX]
\documentclass[fontset=none,a4paper,UTF8]{ctexbook}
\end{lstlisting}
假如您已经指定了 \verb`fontset`~,
那么请改其值为 \verb`none`~.
最后, 在导言区添加
\begin{lstlisting}[language=TeX]
\setmainfont{XITS}
\setCJKmainfont{Source Han Serif CN}
\end{lstlisting}
即可改主要的\gls{eng}的字体为 XITS,
且改主要的\gls{chi}的字体为 Source Han Serif CN (思源宋体).

\vspace*{4ex}

Well. To be continued.

I have been busy with other things in real life.

I will come back.
