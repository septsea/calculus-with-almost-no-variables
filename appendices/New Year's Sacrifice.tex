\chapter{祝福}

\begin{remark*}
    本章是用来测试排版的.
\end{remark*}

\begin{remark*}
    《祝福》是鲁迅创作的短篇小说,
    写于 1924~年 2~月 7~日,
    最初发表于 1924~年 3~月 25~日出版的%
    上海《东方杂志》半月刊第二十一卷第~6~号上,
    后收入小说集《彷徨》.
\end{remark*}

\begin{remark*}
    有意思地, 本作的\gls{eng}翻译的标题是
    \textit{The New Year\textquotesingle{}s Sacrifice}.
    毕竟, 周树人先生在这里写的 ``祝福'' 完全不是
    good wishes (见第~2~段的\emph{被强调的文字}).
\end{remark*}

旧历\footnote{The Chinese lunar calendar.}的年底毕竟最像年底,
村镇上不必说, 就在天空中也显出将到新年的气象来.
灰白色的沉重的晚云中间时时发出闪光, 接着一声钝响, 是送灶的爆竹;
近处燃放的可就更强烈了, 震耳的大音还没有息,
空气里已经散满了幽微的火药香.
我是正在这一夜回到我的故乡鲁镇的.
虽说故乡, 然而已没有家, 所以只得暂寓在鲁四老爷的宅子里.
他是我的本家, 比我长一辈, 应该称之曰 ``四叔'',
是一个讲理学的老监生%
\footnote{The highest institute of learning in the Ching dynasty.}.
他比先前并没有甚么大改变, 单是老了些, 但也还未留胡子,
一见面是寒暄, 寒暄之后说我 ``胖了'',
说我 ``胖了'' 之后即大骂其新党.
但我知道, 这并非借题在骂我:
因为他所骂的还是康有为\footnote{A famous reformist who lived
    from 1858 to 1927 and advocated constitutional monarchy.}.
但是, 谈话是总不投机的了, 于是不多久, 我便一个人剩在书房里.

第二天我起得很迟, 午饭之后, 出去看了几个本家和朋友;
第三天也照样.
他们也都没有甚么大改变, 单是老了些;
\emph{%
    家中却一律忙, 都在准备着 ``祝福''.
    这是鲁镇年终的大典, 致敬尽礼, 迎接福神,
    拜求来年一年中的好运气的.
    杀鸡, 宰鹅, 买猪肉, 用心细细的洗,
    女人的臂膊都在水里浸得通红, 有的还带着绞丝银镯子.
    煮熟之后, 横七竖八的插些筷子在这类东西上,
    可就称为 ``福礼'' 了, 五更天陈列起来, 并且点上香烛,
    恭请福神们来享用, 拜的却只限于男人, 拜完自然仍然是放爆竹.
    年年如此, 家家如此, ——只要买得起福礼和爆竹之类的%
    ——今年自然也如此.%
}
天色愈阴暗了, 下午竟下起雪来, 雪花大的有梅花那么大,
满天飞舞, 夹着烟霭和忙碌的气色, 将鲁镇乱成一团糟.
我回到四叔的书房里时, 瓦楞上已经雪白, 房里也映得较光明,
极分明的显出壁上挂著的朱拓的大 ``寿'' 字, 陈抟%
\footnote{A hermit at the beginning of the tenth century.}%
老祖写的, 一边的对联已经脱落, 松松的卷了放在长桌上,
一边的还在, 道是 ``事理通达心气和平''.
我又无聊赖的到窗下的案头去一翻, 只见一堆似乎未必完全的
《康熙字典》\footnote{A Chinese dictionary compiled under
    the auspices of Emperor Kang Hsi who reigned from 1662
    to 1722.}, 一部《近思录集注》和一部《四书\footnote{%
    Confucian classics.}衬》.
无论如何, 我明天决计要走了.

况且, 一直到昨天遇见祥林嫂的事, 也就使我不能安住.
那是下午, 我到镇的东头访过一个朋友, 走出来, 就在河边遇见她;
而且见她瞪着的眼睛的视线, 就知道明明是向我走来的.
我这回在鲁镇所见的人们中, 改变之大, 可以说无过于她的了:
五年前的花白的头发, 即今已经全白, 会不像四十上下的人;
脸上瘦削不堪, 黄中带黑, 而且消尽了先前悲哀的神色, 仿佛是木刻似的;
只有那眼珠间或一轮, 还可以表示她是一个活物.
她一手提着竹篮.
内中一个破碗, 空的;
一手拄著一支比她更长的竹竿, 下端开了裂:
她分明已经纯乎是一个乞丐了.
我就站住, 豫备她来讨钱.

``你回来了?''
她先这样问.

``是的.''

``这正好.
你是识字的, 又是出门人, 见识得多.
我正要问你一件事——''
她那没有精采的眼睛忽然发光了.

我万料不到她却说出这样的话来, 诧异的站着.

``就是——''
她走近两步, 放低了声音, 极秘密似的切切的说,
``一个人死了之后, 究竟有没有魂灵的?''

我很悚然, 一见她的眼钉着我的, 背上也就遭了芒刺一般,
比在学校里遇到不及豫防的临时考, 教师又偏是站在身旁的时候,
惶急得多了.
对于魂灵的有无, 我自己是向来毫不介意的;
但在此刻, 怎样回答她好呢? 我在极短期的踌躇中, 想,
这里的人照例相信鬼, 然而她, 却疑惑了, ——或者不如说希望:
希望其有, 又希望其无……,
人何必增添末路的人的苦恼, 一为她起见, 不如说有罢.

``也许有罢, ——我想.''
我于是吞吞吐吐的说.

``那么, 也就有地狱了?''

``啊! 地狱?''
我很吃惊, 只得支吾著,
``地狱? ——论理, 就该也有.
——然而也未必, ……谁来管这等事…….''

``那么, 死掉的一家的人, 都能见面的?''

``唉唉, 见面不见面呢? ……''
这时我已知道自己也还是完全一个愚人, 甚么踌躇, 甚么计划,
都挡不住三句问, 我即刻胆怯起来了, 便想全翻过先前的话来,
``那是, ……实在, 我说不清…….
其实, 究竟有没有魂灵, 我也说不清.''

我乘她不再紧接的问, 迈开步便走,
勿勿的逃回四叔的家中, 心里很觉得不安逸.
自己想, 我这答话怕于她有些危险.
她大约因为在别人的祝福时候, 感到自身的寂寞了,
然而会不会含有别的甚么意思的呢? ——或者是有了甚么豫感了?
倘有别的意思, 又因此发生别的事, 则我的答话委实该负若干的责任…….
但随后也就自笑, 觉得偶尔的事, 本没有甚么深意义,
而我偏要细细推敲, 正无怪教育家要说是生著神经病;
而况明明说过 ``说不清'', 已经推翻了答话的全局,
即使发生甚么事, 于我也毫无关系了.

``说不清'' 是一句极有用的话.
不更事的勇敢的少年, 往往敢于给人解决疑问, 选定医生, 万一结果不佳, 大抵反成了怨府, 然而一用这说不清来作结束, 便事事逍遥自在了.
我在这时, 更感到这一句话的必要,
即使和讨饭的女人说话, 也是万不可省的.

但是我总觉得不安, 过了一夜, 也仍然时时记忆起来,
仿佛怀着甚么不祥的豫感, 在阴沉的雪天里, 在无聊的书房里,
这不安愈加强烈了.
不如走罢, 明天进城去.
福兴楼的清炖鱼翅, 一元一大盘, 价廉物美, 现在不知增价了否?
往日同游的朋友, 虽然已经云散,
然而鱼翅是不可不吃的, 即使只有我一个…….
无论如何, 我明天决计要走了.

我因为常见些但愿不如所料, 以为未毕竟如所料的事,
却每每恰如所料的起来, 所以很恐怕这事也一律.
果然, 特别的情形开始了.
傍晚, 我竟听到有些人聚在内室里谈话, 仿佛议论甚么事似的, 但不一会, 说话声也就止了, 只有四叔且走而且高声的说:

``不早不迟, 偏偏要在这时候——这就可见是一个谬种!''

我先是诧异, 接着是很不安, 似乎这话于我有关系.
试望门外, 谁也没有.
好容易待到晚饭前他们的短工来冲茶, 我才得了打听消息的机会.

``刚才, 四老爷和谁生气呢?''
我问.

``还不是和祥林嫂?''
那短工简捷的说.

``祥林嫂? 怎么了?''
我又赶紧的问.

``老了.''

``死了?''
我的心突然紧缩, 几乎跳起来, 脸上大约也变了色,
但他始终没有抬头, 所以全不觉.
我也就镇定了自己, 接着问:

``甚么时候死的?''

``甚么时候? ——昨天夜里, 或者就是今天罢.
——我说不清.''

``怎么死的?''

``怎么死的? ——还不是穷死的?''
他淡然的回答, 仍然没有抬头向我看, 出去了.

然而我的惊惶却不过暂时的事, 随着就觉得要来的事, 已经过去,
并不必仰仗我自己的 ``说不清'' 和他之所谓 ``穷死的'' 的宽慰,
心地已经渐渐轻松;
不过偶然之间, 还似乎有些负疚.
晚饭摆出来了, 四叔俨然的陪着.
我也还想打听些关于祥林嫂的消息,
但知道他虽然读过 ``鬼神者, 二气之良能也''%
\footnote{A Confucian saying.},
而忌讳仍然极多, 当临近祝福时候,
是万不可提起死亡、疾病之类的话的,
倘不得已, 就该用一种替代的隐语,
可惜我又不知道, 因此屡次想问, 而终于中止了.
我从他俨然的脸色上, 又忽而疑他正以为我不早不迟,
偏要在这时候来打搅他, 也是一个谬种, 便立刻告诉他明天要离开鲁镇, 进城去, 趁早放宽了他的心.
他也不很留.
这样闷闷的吃完了一餐饭.

冬季日短, 又是雪天, 夜色早已笼罩了全市镇.
人们都在灯下匆忙, 但窗外很寂静.
雪花落在积得厚厚的雪褥上面, 听去似乎瑟瑟有声, 使人更加感得沉寂.
我独坐在发出黄光的菜油灯下, 想,
这百无聊赖的祥林嫂, 被人们弃在尘芥堆中的,
看得厌倦了的陈旧的玩物, 先前还将形骸露在尘芥里,
从活得有趣的人们看来, 恐怕要怪讶她何以还要存在,
现在总算被无常打扫得干净净了.
魂灵的有无, 我不知道;
然而在现世, 则无聊生者不生,
即使厌见者不见, 为人为己, 也还都不错.
我静听着窗外似乎瑟瑟作响的雪花声,
一面想, 反而渐渐的舒畅起来.

然而先前所见所闻的她的半生事迹的断片,
至此也联成一片了.

她不是鲁镇人.
有一年的冬初, 四叔家里要换女工,
做中人的卫老婆子带她进来了,
头上扎著白头绳, 乌裙, 蓝夹袄, 月白背心,
年纪大约二十六七, 脸色青黄, 但两颊却还是红的.
卫老婆子叫她祥林嫂, 说是自己母家的邻舍,
死了当家人, 所以出来做工了.
四叔皱了皱眉, 四婶已经知道了他的意思, 是在讨厌她是一个寡妇.
但是她模样还周正, 手脚都壮大, 又只是顺着限,
不开一句口, 很像一个安分耐劳的人,
便不管四叔的皱眉, 将她留下了.
试工期内, 她整天的做, 似乎闲著就无聊,
又有力, 简直抵得过一个男子,
所以第三天就定局, 每月工钱五百文.

大家都叫她祥林嫂;
没问她姓甚么, 但中人是卫家山人,
既说是邻居, 那大概也就姓卫了.
她不很爱说话, 别人问了才回答, 答的也不多.
直到十几天之后, 这才陆续的知道她家里还有严厉的婆婆,
一个小叔子, 十多岁, 能打柴了;
她是春天没了丈夫的;
他本来也打柴为生, 比她小十岁%
\footnote{In old China
    it used to be common in country districts for young women
    to be married to boys of ten or eleven.
    The bride\textquotesingle{}s labour could then
    be exploited by her husband\textquotesingle{}s family.}:
大家所知道的就只是这一点.

日子很快的过去了, 她的做工却毫没有懈, 食物不论, 力气是不惜的.
人们都说鲁四老爷家里雇著了女工, 实在比勤快的男人还勤快.
到年底, 扫尘, 洗地, 杀鸡, 宰鹅, 彻夜的煮福礼,
全是一人担当, 竟没有添短工.
然而她反满足, 口角边渐渐的有了笑影, 脸上也白胖了.

新年才过, 她从河边掏米回来时,
忽而失了色, 说刚才远远地看见几个男人在对岸徘徊,
很像夫家的堂伯, 恐怕是正在寻她而来的.
四婶很惊疑, 打听底细, 她又不说.
四叔一知道, 就皱一皱眉, 道:

``这不好.
恐怕她是逃出来的.''

她诚然是逃出来的, 不多久, 这推想就证实了.

此后大约十几天, 大家正已渐渐忘却了先前的事,
卫老婆子忽而带了一个三十多岁的女人进来了, 说那是祥林嫂的婆婆.
那女人虽是山里人模样, 然而应酬很从容, 说话也能干,
寒暄之后, 就赔罪, 说她特来叫她的儿媳回家去,
因为开春事务忙, 而家中只有老的和小的, 人手不够了.

``既是她的婆婆要她回去, 那有甚么话可说的呢?'' 四叔说.
于是算清了工钱, 一共一千七百五十文,
她全存在主人家, 一文也还没有用, 便都交给她的婆婆.
那女人又取了衣服, 道过谢, 出去了.
其时已经是正午.

``阿呀, 米呢? 祥林嫂不是去淘米的么? ……''
好一会, 四婶这才惊叫起来.
她大约有些饿, 记得午饭了.

于是大家分头寻淘箩.
她先到厨下, 次到堂前, 后到卧房, 全不见掏箩的影子.
四叔踱出门外, 也不见, 一直到河边,
才见平平正正的放在岸上, 旁边还有一株菜.

看见的人报告说, 河里面上午就泊了一只白篷船,
篷是全盖起来的, 不知道甚么人在里面, 但事前也没有人去理会他.
待到祥林嫂出来掏米, 刚刚要跪下去,
那船里便突然跳出两个男人来, 像是山里人,
一个抱住她, 一个帮着, 拖进船去了.
祥林嫂还哭喊了几声, 此后便再没有甚么声息, 大约给用甚么堵住了罢.
接着就走上两个女人来, 一个不认识, 一个就是卫婆于.
窥探舱里, 不很分明, 她像是捆了躺在船板上.

``可恶!
然而…….''
四叔说.

这一天是四婶自己煮中饭;
他们的儿子阿牛烧火.

午饭之后, 卫老婆子又来了.

``可恶!'' 四叔说.

``你是甚么意思? 亏你还会再来见我们.''
四婶洗著碗, 一见面就愤愤的说,
``你自己荐她来, 又合伙劫她去,
闹得沸反盈天的, 大家看了成个甚么样子?
你拿我们家里开玩笑么?''

``阿呀阿呀, 我真上当.
我这回, 就是为此特地来说说清楚的.
她来求我荐地方, 我那里料得到是瞒着她的婆婆的呢.
对不起, 四老爷, 四太太.
总是我老发昏不小心, 对不起主顾.
幸而府上是向来宽洪大量, 不肯和小人计较的.
这回我一定荐一个好的来折罪…….''

``然而…….'' 四叔说.

于是祥林嫂事件便告终结, 不久也就忘却了.

只有四嫂, 因为后来雇用的女工,
大抵非懒即馋, 或者馋而且懒, 左右不如意,
所以也还提起祥林嫂.
每当这些时候, 她往往自言自语的说,
``她现在不知道怎么样了?''
意思是希望她再来.
但到第二年的新正, 她也就绝了望.

新正将尽, 卫老婆子来拜年了, 已经喝得醉醺醺的,
自说因为回了一趟卫家山的娘家, 住下几天, 所以来得迟了.
她们问答之间, 自然就谈到祥林嫂.

``她么?''
卫若婆子高兴的说,
``现在是交了好运了.
她婆婆来抓她回去的时候,
是早已许给了贺家坳的贺老六的,
所以回家之后不几天,
也就装在花轿里抬去了.''

``阿呀, 这样的婆婆! ……''
四婶惊奇的说.

``阿呀, 我的太太!
你真是大户人家的太太的话.
我们山里人, 小户人家, 这算得甚么?
她有小叔子, 也得娶老婆.
不嫁了她, 那有这一注钱来做聘礼?
他的婆婆倒是精明强干的女人呵, 很有打算, 所以就将地嫁到里山去.
倘许给本村人, 财礼就不多;
惟独肯嫁进深山野坳里去的女人少, 所以她就到手了八十千.
现在第二个儿子的媳妇也娶进了, 财礼花了五十,
除去办喜事的费用, 还剩十多千.
吓, 你看, 这多么好打算? ……''

``祥林嫂竟肯依? ……''

``这有甚么依不依.
——闹是谁也总要闹一闹的, 只要用绳子一捆, 塞在花轿里,
抬到男家, 捺上花冠, 拜堂, 关上房门, 就完事了.
可是祥林嫂真出格, 听说那时实在闹得利害,
大家还都说大约因为在念书人家做过事, 所以与众不同呢.
太太, 我们见得多了:
回头人出嫁, 哭喊的也有, 说要寻死觅活的也有,
抬到男家闹得拜不成天地的也有, 连花烛都砸了的也有.
样林嫂可是异乎寻常,
他们说她一路只是嚎, 骂, 抬到贺家坳, 喉咙已经全哑了.
拉出轿来, 两个男人和她的小叔子使劲的捺住她也还拜不成夭地.
他们一不小心, 一松手, 阿呀阿弥陀佛,
她就一头撞在香案角上, 头上碰了一个大窟窿,
鲜血直流, 用了两把香灰, 包上两块红布还止不住血呢.
直到七手八脚的将她和男人反关在新房里, 还是骂,
阿呀呀, 这真是…….''
她摇一摇头, 顺下眼睛, 不说了.

``后来怎么样呢?''
四婢还问.

``听说第二天也没有起来.''
她抬起眼来说.

``后来呢?''

``后来? ——起来了.
她到年底就生了一个孩子, 男的, 新年就两岁%
\footnote{It was the custom in China
    to reckon a child as one year old at birth,
    and to add another year to his age as New Year.}了.
我在娘家这几天, 就有人到贺家坳去,
回来说看见他们娘儿俩, 母亲也胖, 儿子也胖;
上头又没有婆婆, 男人所有的是力气, 会做活;
房子是自家的.
——唉唉, 她真是交了好运了.''

从此之后, 四婶也就不再提起祥林嫂.

但有一年的秋季, 大约是得到祥林嫂好运的消息之后的又过了两个新年,
她竟又站在四叔家的堂前了.
桌上放著一个荸荠式的圆篮, 檐下一个小铺盖.
她仍然头上扎著白头绳, 乌裙, 蓝夹袄, 月白背心, 脸色青黄,
只是两颊上已经消失了血色, 顺着眼,
眼角上带些泪痕, 眼光也没有先前那样精神了.
而且仍然是卫老婆子领着, 显出慈悲模样, 絮絮的对四婶说:

``……这实在是叫作 `天有不测风云',
她的男人是坚实人, 谁知道年纪轻轻, 就会断送在伤寒上?
本来已经好了的, 吃了一碗冷饭, 复发了.
幸亏有儿子;
她又能做, 打柴摘茶养蚕都来得, 本来还可以守着,
谁知道那孩子又会给狼衔去的呢?
春天快完了, 村上倒反来了狼, 谁料到?
现在她只剩了一个光身了.
大伯来收屋, 又赶她.
她真是走投无路了, 只好来求老主人.
好在她现在已经再没有甚么牵挂,
太太家里又凄巧要换人,
所以我就领她来——%
我想, 熟门熟路, 比生手实在好得多……''

``我真傻, 真的,''
祥林嫂抬起她没有神采的眼睛来, 接着说.
``我单知道下雪的时候野兽在山坳里没有食吃, 会到村里来;
我不知道春天也会有.
我一清早起来就开了门, 拿小篮盛了一篮豆,
叫我们的阿毛坐在门槛上剥豆去.
他是很听话的, 我的话句句听;
他出去了.
我就在屋后劈柴, 掏米, 米下了锅, 要蒸豆.
我叫阿毛, 没有应,
出去口看, 只见豆撒得一地, 没有我们的阿毛了.
他是不到别家去玩的;
各处去一问, 果然没有.
我急了, 央人出去寻.
直到下半天, 寻来寻去寻到山坳里, 看见刺柴上桂著一只他的小鞋.
大家都说, 糟了, 怕是遭了狼了.
再进去;
他果然躺在草窠里, 肚里的五脏已经都给吃空了,
手上还紧紧的捏著那只小篮呢.
……''
她接着但是呜咽, 说不出成句的话来.

四婶起刻还踌踌, 待到听完她自己的话, 眼圈就有些红了.
她想了一想, 便教拿圆篮和铺盖到下房去.
卫老婆子仿佛卸了一肩重相似的嘘一口气,
祥林嫂比初来时候神气舒畅些, 不待指引, 自己驯熟的安放了铺盖.
她从此又在鲁镇做女工了.

大家仍然叫她祥林嫂.

然而这一回, 她的境遇却改变得非常大.
上工之后的两三天, 主人们就觉得她手脚已没有先前一样灵活,
记性也坏得多, 死尸似的脸上又整日没有笑影,
四婶的口气上, 已颇有些不满了.
当她初到的时候, 四叔虽然照例皱过眉,
但鉴于向来雇用女工之难, 也就并不大反对,
只是暗暗地告诫四姑说,
这种人虽然似乎很可怜, 但是败坏风俗的,
用她帮忙还可以, 祭祀时候可用不着她沾手,
一切饭菜, 只好自已做,
否则, 不干不净, 祖宗是不吃的.

四叔家里最重大的事件是祭祀,
祥林嫂先前最忙的时候也就是祭祀,
这回她却清闲了.
桌子放在堂中央, 系上桌帏,
她还记得照旧的去分配酒杯和筷子.

``祥林嫂, 你放著罢!
我来摆.''
四婶慌忙的说.

她讪讪的缩了手, 又去取烛台.

``祥林嫂, 你放著罢!
我来拿.'' 四婶又慌忙的说.

她转了几个圆圈, 终于没有事情做, 只得疑惑的走开.
她在这一天可做的事是不过坐在灶下烧火.

镇上的人们也仍然叫她祥林嫂, 但音调和先前很不同;
也还和她讲话, 但笑容却冷冷的了.
她全不理会那些事, 只是直着眼睛, 和大家讲她自己日夜不忘的故事:

``我真傻, 真的,''
她说,
``我单知道雪天是野兽在深山里没有食吃, 会到村里来;
我不知道春天也会有.
我一大早起来就开了门, 拿小篮盛了一篮豆,
叫我们的阿毛坐在门槛上剥豆去.
他是很听话的孩子, 我的话句句听;
他就出去了.
我就在屋后劈柴, 淘米, 米下了锅, 打算蒸豆.
我叫 `阿毛!', 没有应.
出去一看, 只见豆撒得满地, 没有我们的阿毛了.
各处去一向, 都没有.
我急了, 央人去寻去.
直到下半天, 几个人寻到山坳里, 看见刺柴上挂著一只他的小鞋.
大家都说, 完了, 怕是遭了狼了;
再进去;
果然, 他躺在草窠里, 肚里的五脏已经都给吃空了,
可怜他手里还紧紧的捏著那只小篮呢.
……''
她于是淌下眼泪来, 声音也呜咽了.

这故事倒颇有效,
男人听到这里, 往往敛起笑容, 没趣的走了开去;
女人们却不独宽恕了她似的, 脸上立刻改换了鄙薄的神气,
还要陪出许多眼泪来.
有些老女人没有在街头听到她的话, 便特意寻来,
要听她这一段悲惨的故事.
直到她说到呜咽, 她们也就一齐流下那停在眼角上的眼泪,
叹息一番, 满足的去了, 一面还纷纷的评论著.

她就只是反复的向人说她悲惨的故事, 常常引住了三五个人来听她.
但不久, 大家也都听得纯熟了,
便是最慈悲的念佛的老太太们, 眼里也再不见有一点泪的痕迹.
后来全镇的人们几乎都能背诵她的话, 一听到就烦厌得头痛.

``我真傻, 真的,''
她开首说.

``是的, 你是单知道雪天野兽在深山里没有食吃, 才会到村里来的.''
他们立即打断她的话, 走开去了.

她张著口怔怔的站着, 直着眼睛看他们,
接着也就走了, 似乎自己也觉得没趣.
但她还妄想, 希图从别的事, 如小篮, 豆, 别人的孩子上,
引出她的阿毛的故事来.
倘一看见两三岁的小孩子, 她就说:

``唉唉, 我们的阿毛如果还在, 也就有这么大了……''

孩子看见她的眼光就吃惊, 牵着母亲的衣襟催她走.
于是又只剩下她一个, 终于没趣的也走了,
后来大家又都知道了她的脾气, 只要有孩子在眼前,
便似笑非笑的先问她, 道:

``祥林嫂, 你们的阿毛如果还在, 不是也就有这么大了么?''

她未必知道她的悲哀经大家咀嚼赏鉴了许多天,
早已成为渣滓, 只值得烦厌和唾弃;
但从人们的笑影上, 也仿佛觉得这又冷又尖, 自己再没有开口的必要了.
她单是一瞥他们, 并不回答一句话.

鲁镇永远是过新年, 腊月二十以后就火起来了.
四叔家里这回须雇男短工, 还是忙不过来,
另叫柳妈做帮手, 杀鸡, 宰鹅;
然而柳妈是 ``善女人'', 吃素、不杀生的, 只肯洗器皿.
祥林嫂除烧火之外, 没有别的事, 却闲著了, 坐着只看柳妈洗器皿.
微雪点点的下来了.

``唉唉, 我真傻,''
祥林嫂看了天空, 叹息著, 独语似的说.

``祥林嫂, 你又来了.''
柳妈不耐烦的看着她的脸, 说.
``我问你:
你额角上的伤痕, 不就是那时撞坏的么?''

``晤晤.''
她含糊的回答.

``我问你:
你那时怎么后来竟依了呢?''

``我么? ……'',

``你呀.
我想:
这总是你自己愿意了, 不然…….''

``阿阿, 你不知道他力气多么大呀.''

``我不信.
我不信你这么大的力气, 真会拗他不过.
你后来一定是自己肯了, 倒推说他力气大.''

``啊啊, 你……你倒自己试试着.'' 她笑了.

柳妈的打皱的脸也笑起来,
使她蹙缩得像一个核桃, 干枯的小眼睛一看祥林嫂的额角, 又钉住她的眼.
祥林嫂似很局促了, 立刻敛了笑容, 旋转眼光, 自去看雪花.

``祥林嫂, 你实在不合算.''
柳妈诡秘的说.
``再一强, 或者索性撞一个死, 就好了.
现在呢, 你和你的第二个男人过活不到两年, 倒落了一件大罪名.
你想, 你将来到阴司去, 那两个死鬼的男人还要争, 你给了谁好呢?
阎罗大王只好把你锯开来, 分给他们.
我想, 这真是……''

她脸上就显出恐怖的神色来, 这是在山村里所未曾知道的.

``我想, 你不如及早抵当.
你到土地庙里去捐一条门槛, 当作你的替身,
给千人踏, 万人跨, 赎了这一世的罪名, 免得死了去受苦.''

她当时并不回答甚么话, 但大约非常苦闷了,
第二天早上起来的时候, 两眼上便都围着大黑圈.
早饭之后, 她便到镇的西头的土地庙里去求捐门槛,
庙祝起初执意不允许, 直到她急得流泪, 才勉强答应了.
价目是大钱十二千.
她久已不和人们交口, 因为阿毛的故事是早被大家厌弃了的;
但自从和柳妈谈了天, 似乎又即传扬开去,
许多人都发生了新趣味, 又来逗她说话了.
至于题目, 那自然是换了一个新样, 专在她额上的伤疤.

``祥林嫂, 我问你:
你那时怎么竟肯了?'' 一个说.

``唉, 可惜, 白撞了这-下.''
一个看着她的疤, 应和道.

她大约从他们的笑容和声调上, 也知道是在嘲笑她,
所以总是瞪着眼睛, 不说一句话, 后来连头也不回了.
她整日紧闭了嘴唇, 头上带着大家以为耻辱的记号的那伤痕,
默默的跑街, 扫地, 洗菜, 淘米.
快够一年, 她才从四婶手里支取了历来积存的工钱,
换算了十二元鹰洋, 请假到镇的西头去.
但不到一顿饭时候, 她便回来, 神气很舒畅,
眼光也分外有神, 高兴似的对四婶说, 自己已经在土地庙捐了门槛了.

冬至的祭祖时节, 她做得更出力, 看四婶装好祭品,
和阿牛将桌子抬到堂屋中央, 她便坦然去拿酒杯和筷子.

``你放著罢, 祥林嫂!'' 四婶慌忙大声说.

她像是受了炮烙似的缩手, 脸色同时变作灰黑,
也不再去取烛台, 只是失神的站着.
直到四叔上香的时候, 教她走开, 她才走开.
这一回她的变化非常大,
第二天, 不但眼睛窈陷下去, 连精神也更不济了.
而且很胆怯, 不独怕暗夜, 怕黑影,
即使看见人, 虽是自己的主人, 也总惴惴的,
有如在白天出穴游行的小鼠, 否则呆坐着, 直是一个木偶人.
不半年, 头发也花白起来了, 记性尤其坏, 甚而至于常常忘却了去淘米.

``祥林嫂怎么这样了? 倒不如那时不留她.''
四婶有时当面就这样说, 似乎是警告她.

然而她总如此, 全不见有伶俐起来的希望.
他们于是想打发她走了, 教她回到卫老婆子那里去.
但当我还在鲁镇的时候, 不过单是这样说;
看现在的情状, 可见后来终于实行了.
然而她是从四叔家出去就成了乞丐的呢,
还是先到卫老婆子家然后再成乞丐的呢?
那我可不知道.

我给那些因为在近旁而极响的爆竹声惊醒,
看见豆一般大的黄色的灯火光,
接着又听得毕毕剥剥的鞭炮,
是四叔家正在 ``祝福'' 了;
知道已是五更将近时候.
我在蒙胧中, 又隐约听到远处的爆竹声联绵不断,
似乎合成一天音响的浓云, 夹着团团飞舞的雪花, 拥抱了全市镇.
我在这繁响的拥抱中, 也懒散而且舒适,
从白天以至初夜的疑虑, 全给祝福的空气一扫而空了,
只觉得天地圣众歆享了牲醴和香烟, 都醉醺醺的在空中蹒跚,
豫备给鲁镇的人们以无限的幸福.

\begin{flushright}
    一九二四年二月七日
\end{flushright}
