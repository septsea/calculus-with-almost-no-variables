\documentclass[fontset=fandol]{ctexbook}

\title{这是书的标题}
\author{这是作者}
\date{这是日期}

\begin{document}% 正式开始文档.

\frontmatter

\maketitle

\tableofcontents

\listoftables

\chapter{前言}

我想向乳胶问好.

\mainmatter

\part{甲}

\chapter{我向乳胶问好}

您好, 乳胶\footnote{胡话里, 乳胶就是 ``latex''.}!
或许, 我应该叫您的原名, \LaTeX{}.

I want%
    to learn
        100\% about \LaTeX{}.

\section{指数与三角}

据说, 指数 $\exp$ 跟三角 $\cos$, $\sin$ 间有这样的联系:
\begin{equation}\label{eq:Euler}
    \mathrm{e}^{\mathrm{i} x} = \cos {x} + \mathrm{i} \sin {x}.
\end{equation}
不过, 这到底是什么意思呢?

似乎还是要给定义的.
指数函数的定义是
\[
    \mathrm{e}^{z} = 1 + \frac{z^1}{1!} + \frac{z^2}{2!}
    + \dots + \frac{z^n}{n!} + \dots.
\]
$\cos$ 的定义是
\[
    \cos {z}
    = \frac{\mathrm{e}^{\mathrm{i} z}
    + \mathrm{e}^{-\mathrm{i} z}}{2},
\]
而 $\sin$ 的定义是
\[
    \sin {z}
    = \frac{\mathrm{e}^{\mathrm{i} z}
    - \mathrm{e}^{-\mathrm{i} z}}{2\mathrm{i}}.
\]

所以, 根据定义,
\[
    \cos {x} + \mathrm{i} \sin {x}
    = \frac{\mathrm{e}^{\mathrm{i} x}
    + \mathrm{e}^{-\mathrm{i} x}}{2}
    + \frac{\mathrm{e}^{\mathrm{i} x}
    - \mathrm{e}^{-\mathrm{i} x}}{2}
    = \mathrm{e}^{\mathrm{i} x}.
\]

\section{算学家: Leonhard Euler}

据说, 算学家 Leonhard Euler 早就发现式~(\ref{eq:Euler}).
今日, 大家一般都叫它 Euler 公式.

Euler 是谁?
他是 18~世纪的瑞士\footnote{一个欧洲国家.}算学家.\\
1707~年 4~月 15~日出生于瑞士巴塞尔;\\
1783~年 9~月 18~日逝世于俄国\footnote{一个欧洲国家.}圣彼得堡.

事实上, 有很多 Euler 公式;\par
不过, 我就不多说了.

\clearpage% 分页.

可用 \verb/\clearpage/ 手动换页.

顺便, 也总算看到奇数页的非空页眉了\footnote{也就是说,
    奇数页的非空页眉的左侧是节标题, 右侧是页码.}.

\chapter{不知道写什么了}

\section{摆了}

我问不下去了啊\footnote{因为真没什么东西可问了.}.

我暂且列出一些术语罢 (表~\ref{tab:Terms}).

\begin{table}[h!]
    \centering
    \caption{术语表}\label{tab:Terms}
    \begin{tabular}{l l}
        \hline
        胡话   & English       \\
        土话   & Chinese       \\
        乳胶   & \LaTeX{}      \\
        算学   & mathematics   \\
        算学家 & mathematician \\
        \hline
    \end{tabular}
\end{table}

没别的了.

\part{乙}

\chapter{章}

\section{节}

\subsection{小节}

\subsubsection{小小节}

差不多了.
其实还有 \verb|paragraph| 跟 \verb-subparagraph-,
但似乎也没什么用.
很多书一般不会分这么多层罢\footnote{应该不会罢.}.

\paragraph{段}

您好. 我紧跟着 ``段''\footnote{确实.}.

\subparagraph{小段}

您好. 我也紧跟着 ``小段''. 我还缩进了.

\subparagraph{段小}

您好. 我也紧跟着 ``段小''. 我还缩进了.

\clearpage

就这样罢.

\appendix

\chapter{乳胶的历史}

\section{什么是乳胶?}

乳胶 (胡话: \LaTeX{}) 是
Leslie Lamport\footnote{一个美国人.}
作出的一种排版应用.

\backmatter

\chapter{后记}

麻雀虽小, 五脏俱全.

\end{document}% 结束文档.
