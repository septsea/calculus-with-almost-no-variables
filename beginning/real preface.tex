\chapter{序}

嘛; 听我说罢.
我告诉您为什么序后还有序.

我大概是在 2022~年 4~月中旬完成了本书.
当时, 它的副标题还是《78~分快速上手》;
这意味着, 当时, 本书的正文也就只有 78~页.
不过, 读者们似乎不是特别喜欢本书.
有一位读者花了大约 3~分看完了本书;
这确实不是什么好事.
他没说什么; 不过我也能看出, 本书不吸引他.
不少读者花了几十分读本书;
当然了, 大家也都说,
未能体会到什么本书讨论的对象 ``实用'' ``优越''.
我从来没说 ``实用'';
我也不怎么说它 ``优越''
(我也只在前一个序里说 ``无变量的微积分在某些地方确实是有优势的'').
我也承认, 本书的本质跟传统的微积分的教程的本质没有什么不同.
但, 我觉得我在本书使用的记号有条理;
我不觉得本书没有任何价值.
不过, 我说本书有价值是无用的;
\emph{您}觉得本书对\emph{您}有价值, 那才是有价值.
毕竟, 这里的 ``有价值'' 跟主观的感受有关罢.

所以, 本书转型了.
原本, 本书只有\gls{math}的东西;
现在, 本书还讨论怎么用\gls{latex}排版.
我想, 这至少会吸引一些想学\gls{latex}的人罢.
当然, 本书不是从\emph{零}教您学\gls{latex};
您可视本书为\gls{latex}参考材料,
从本书 (或其代码) 里挑出您感兴趣的东西.

所以, 前一个序是 ``假序'', 而本序才是 ``真序''.

就说这么多罢.

\begin{flushright}
    \theauthor\\
    \thedate
\end{flushright}
