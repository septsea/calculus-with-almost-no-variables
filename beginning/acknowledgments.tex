\chapter{Acknowledgments}

I\marginpar{%
    This shows how a margin note looks.
    Margin notes are not often seen in Chinese books.
    I do see them in manuscripts, though.%
}
owe a huge debt
to many organizations and people
that directly or indirectly provided me
with their assistance.

I thank Jakub Marian for his easy-to-comprehend articles
which discuss calculus with almost no variables
and other related topics.
I thank Karl Menger for a more detailed discussion.
I thank Constantin Carath\'eodory
for a new and handy definition of derivatives.
I thank Stephen Kuhn for promoting the new method.
I thank Ruan Long\-pei and Wu Chang\-qing
for pointing out an easily ignored ``bug''
in high school mathematics.
I thank Walter Rudin, Mei Jia\-qiang and Zhang Zhu\-sheng
for their excellent mathematical analysis textbooks.

I\marginpar{%
    I can even say that
    I will die without these applications.
}
thank the Comprehensive \hologo{TeX} Archive Network.
I thank \hologo{TeX} Live.
I thank all packages used in the book,
which I have listed in the bibliography.
I would not be able to write a book easily and freely
without them.
I thank Visual Studio Code.
I thank Vim (or more precisely, the VS\-Code\-Vim extension).
I would not be able to type so smoothly without them.
I thank AutoHotKey,
which I use mainly for defining my own keybindings.
I thank Emacs, which can be used to type Chinese characters.
I thank Firefox, the open-source browser
which I have been using for years.
I thank Windows 11 Pro Insider Preview,
the operating system which I have been using
for approximately a year
and which has fewer unpleasing bugs
compared with the stable releases.
I thank \hologo{LaTeX} Stack Exchange,
which is full of \hologo{LaTeX} treasures.
I thank myself for spending
a considerable amount of time on English
so that I do not have much difficulty
in reading package documentations
or searching what I need on the World Wide Web.

I thank Sukera\-Sparo for creating educational visual novels,
\textit{The Expression Amrilato}
and \textit{Distant Memora\^jo}.
I really had a good time.
I also picked up some Esperanto
(or more precisely, \textit{Juliamo}).
I am still able to count in the language
made by Ludwig Lazarus Zamenhof
(\textit{nul}, \textit{unu}, \textit{du}, \textit{tri},
\textit{kvar}, \textit{kvin}, \textit{ses}, \textit{sep},
\textit{ok}, \textit{na\^u}, \textit{dek}, \textit{dek unu},
\textit{dek du}, \textit{dek tri},
\textit{dek kvar}, \textit{dek kvin},
etc.);
I am still able to say \textit{Dankon!}\ to thank others
or \textit{Gratulon!}\ to congratulate others;
however, I do not have many opportunities
to use it as frequently as I use English.
That is really a pity.

There must still be many organizations and people
that helped me but are not mentioned in my acknowledgments.
I apologize for not listing them here,
since I did not keep track of everything.

Finally, I give my special thanks to someone
who is ``unable to name anything''
for the Hosimiya Sio decoration
and some other presents in the three-dimensional world,
which really brightened me when I felt depressed.

% Temporarily insert an English name and date.
\ctexset{today=old}
\begin{flushright}
    Septsea\\
    \today
\end{flushright}
\ctexset{today=small}
