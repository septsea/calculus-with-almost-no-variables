\setglossarystyle{long}

\setglossarypreamble[lore]{%
    \glsresetentrycounter%

    \begin{remark*}
        本章是用来测试术语表的排版的.
    \end{remark*}

    \begin{remark*}
        所谓 Anglish, 就是这么一种\gls{eng}变体:
        它尽量挑\gls{eng}自己的词, 而不是挑从别的话里借来的词.
        比方说,
        用 ``正常的\gls{eng}'' 表达一个世纪 (也就是 100~年),
        就是 a century;
        不过, \gls{eng}的 century 是%
        从拉丁话 \textit{centuria} 来的.
        那么, Anglish 会怎么表达此事呢?
        可以说 a hundred years, 也可说 five score years
        (\gls{eng}的 score 就是 20).
    \end{remark*}

    \begin{remark*}
        \textit{Lores} are what are know as \textit{studies}
        in the English tung (language).
        There are a great many of them with names,
        but almost all of these names come from
        either the Latin or Greek tung.

        Below is a list of lores given first in English,
        and then in Anglish.

        The honewordy hue (adjective form)
        of the nounword (noun) \textit{lore} is \textit{lorely},
        that of \textit{frood} is \textit{frodly},
        while that of \textit{craft} is \textit{crafty}.
    \end{remark*}%
}
\printnoidxglossary[sort=standard,type=lore]
