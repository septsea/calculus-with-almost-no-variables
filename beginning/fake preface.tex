\chapter[序]{序\footnote{The preface was written
      on the seventeenth of April in 2022.
      Too many things happened,
      so I just slightly modified it
      and wrote a ``real preface.''}}

本书的标题是《\thetitle{}》.
您按字面意思理解此标题就好;
本书讨论的微积分并不是没有变量, 而是减少了变量的使用.
本书的 ``微积分'' 跟 ``分析学'' 不一样.
我姑且这么描述:
分析学更偏向理论与理论间的联系
(如极限、连续、导数、积分等概念的联系),
而微积分更偏向具体的计算
(您当然可以认为 ``微积分'' 就是分析学;
这样的话,
本书的标题就应该是《几乎无变量地计算导数、不定积分、积分》).
本书是一本算普读物 (\gls{math}普及读物).
本书并\emph{不是}从\emph{零}教您微积分
(假如我要写这样的书,
那我可能要更多时间与更多力气大改现有的微积分符号);
相反, 本书假定您会 (最基本的) 微积分.
这样, 我就可以专心展现无变量的微积分演算是什么样的.
您可以对比我的书的跟传统的微积分教材的
(或高等\gls{math}教材, 也可以是\gls{math}分析教材)
计算过程.
这样, 您可以看到这种 (几乎) 无变量的微积分%
在某些地方确实是有优势的.

本书的副标题是《\thesubtitle{}》.
其实, 您不必太在意这个副标题里的
``\pageref{calculus:LastPage}~分'';
这只是一个\gls{joke}.
事实上, 本书的正文就只有 \pageref{calculus:LastPage}~页.
我假定您至多用 1~分 (即 60~秒) 看 1~页.
这么看来, 您至多用 \pageref{calculus:LastPage}~分%
就可以了解最基本的微积分.
不过, 认真地, 若我至多用 1~分看 1~页\emph{\gls{math}书},
我可能学不到什么东西.

我不是这本小书欲讨论的对象的创始人.
一位美籍奥地利裔\gls{mathematician} Karl Menger
在 1949~年发表了名为
\textit{Are variables necessary in calculus?}\ 的文章.
一位捷克的数据科学家、语言学家、地理学家与音乐人
Jakub Marian 在 2014~年又提到了这个话题.
我在 2022~年~3~月也独立地搞出了一些东西.
不过, 我菜, 只搞出了 ``几乎无变量的一元微积分''.
当我想写这本小书时, 我才开始查阅文献.
不出意外, 我查到了一些资料
(不过并不是很多, 因为跟我的个人计算机焊接的互联网上的资源有限).
我仔细地阅这些资料, 并对自己的记号作出了一些改进.
我在参考文献里列出了无变量的微积分的文献, 您可以去看一看
(毕竟我不能很好地用文字表达我的想法).

相信大家都学过函数.
在初中\gls{math}里, 我们用变量定义函数.
下面是湘教版八年级下册的\gls{math}课本的定义.

\begin{definition*}
    在讨论的问题中,
    称取值会发生变化的量为\emph{变量},
    称取值固定不变的量为\emph{常量} (或\emph{常数}).
\end{definition*}

\begin{definition*}
    一般地,
    如果变量 $y$ 随着变量 $x$ 而变化,
    并且对于 $x$ 取的每一个值,
    $y$ 都有唯一的一个值与它对应,
    那么称 $y$ 是 $x$ 的\emph{函数},
    记作 $y = f(x)$.
    这里的 $f(x)$ 是\gls{eng} a function of $x$
    (\gls{chi}: $x$ 的函数) 的简记.
    这时叫 $x$ 作\emph{自变量},
    叫 $y$ 作\emph{因变量}.
    对于自变量 $x$ 取的每一个值 $a$,
    称因变量 $y$ 的对应值为\emph{函数值},
    并记其作 $f(a)$.
\end{definition*}

这个定义, 虽不是很严谨, 但很形象.
至少, 刚接触 ``函数'' 的人会对函数有比较形象的认识.
早期的\gls{mathematician}就是用 ``这种函数'' 讨论微积分的.
不过, 随着\gls{math}的发展,
\gls{mathematician}需要对\gls{math}对象有严格的阐述.
函数也不例外.
1914~年, 德国\gls{mathematician} Felix Hausdorff 在他的
\textit{Grundz\"uge der Mengenlehre} 里用 ``有序对'' 定义函数
(在本书, 我也会这么定义函数).
这种定义当然避开了非\gls{math}话 ``变量'' ``对应''.
不过, 更严谨地看, ``有序对'' 是什么?
能不能用更基础的东西定义它?
1921~年, 波兰\gls{mathematician} Kazimierz Kuratowski
在他的文章
\textit{%
    Sur la notion de l\textquotesingle{}ordre
    dans la Th\'eorie des Ensembles%
}
里定义 $(a, b)$ 为 $\{ \{a\}, \{a, b\}\}$.
于是, 可以\emph{证明},
\begin{align*}
    (a, b) = (c, d) \iff \text{$a = c$ 且 $b = d$}.
\end{align*}
这样, Hausdorff 的定义就更完美了.

% 我并不打算在本书严谨地讨论集的理论, 所以我就提到这里.

尽管函数有现代的定义,
函数也有现代的、不带变量的记号 $f$ (而不是 $f(x)$),
可我们在进行微积分计算时,
还是用带变量的记号进行计算.
具体地, 我们计算导数时, 用的记号是
\begin{align*}
    f^{\prime} (x) \quad \text{或} \quad \frac{\mathrm{d}}{\mathrm{d}x} {f(x)};
\end{align*}
我们计算不定积分时, 用的记号是
\begin{align*}
    \int {f(x) \,\mathrm{d}x};
\end{align*}
我们计算积分时, 用的记号是
\begin{align*}
    \int_{a}^{b} {f(x) \,\mathrm{d}x}.
\end{align*}

请允许我暂时跑题.
我并没有说这些记号不好.
相反, 这些记号十分经典, 经得起时间与\gls{mathematician}的考验.
我自己初学微积分 (与\gls{math}分析) 时, 就是用这套经典记号的.
比如, 可形象地写求导数的链规则为
\begin{align*}
    \frac{\mathrm{d}y}{\mathrm{d}x}
    = \frac{\mathrm{d}y}{\mathrm{d}u} \cdot \frac{\mathrm{d}u}{\mathrm{d}x},
\end{align*}
这里 $u = f(x)$, $y = g(u) = g(f(x))$.
视导数为 ``变率'', 那这就是在说,
$y$ 关于 $x$ 的变率%
等于 $y$ 关于 $u$ 的变率%
与 $u$ 关于 $x$ 的变率的积.
很形象吧?
假设 A, B, C 三人在直线跑道上匀速前进.
A 的速率是 B 的速率的 $\frac{11}{10}$
(也就是说, A 比 B 快 $\frac{1}{10}$),
而 B 的速率是 C 的速率的 $\frac{9}{10}$
(也就是说, B 比 C 慢 $\frac{1}{10}$),
那么 A 的速率是 C 的速率的 $\frac{99}{100}$;
这就是二个比的积.

回到正题.
我们已经看到, 我们通用的微积分记号带着朴素的函数思想.
此现象让我好奇.
我就想:
``有没有不要变量的微积分?
或者说, 有没有几乎不要变量的微积分?''
我认真思考了几日.
至少, 我已经习惯用 $\mathrm{D}$ 表示求导,
所以导数似乎不是什么问题.
比方说,
$\operatorname{D} \mathrm{exp} = \mathrm{exp}$,
$\operatorname{D} \mathrm{cos} = -\mathrm{sin}$,
$\operatorname{D} \mathrm{sin} = \mathrm{cos}$.
不过,
当我想表达 $\operatorname{D} \mathrm{ln}$ 时,
我意识到了一个重要的问题:
``已知 $\operatorname{D} \ln {x} = 1/x$.
左边的 $\ln {x}$ 就是 $\mathrm{ln}$,
可右边的 $1/x$ 应该是什么?''
想起\gls{eng}里, reciprocal 是倒数的意思, 我就定义
\begin{align*}
    \text{$\mathrm{rec}$:} \quad
    \mathbb{R} \setminus \{ 0 \} & \to \mathbb{R} \setminus \{ 0 \}, \\
    x                            & \mapsto \frac{1}{x}.
\end{align*}
这样, 我就可以写
$\operatorname{D} \mathrm{ln} = \mathrm{rec}$.
不过, 我还是没法好好地表示
$\operatorname{D} \mathrm{arcsin}$
跟
$\operatorname{D} \mathrm{arctan}$.
我这时才意识到,
因为在微积分里,
有名的 (是 named, 而不是 well-known 或 famous) 函数不够多,
所以我想表达普普通通的导数都要自己起名字.
不至于碰到一个函数就起名字吧?
所以, 我定义了所谓的 ``什么也不干'' 的函数
\begin{align*}
    \text{$\mathrm{fdn}$:} \quad
    \mathbb{R} & \to \mathbb{R}, \\
    x          & \mapsto x
\end{align*}
($\mathrm{fdn}$ 乃 the function that does nothing 之略).
这样, 再利用函数的运算,
我总算能无变量地写出基本的求导公式了.

我随后又作出了无变量不定积分与无变量积分的理论.
不过, 我写不下去了
(没作出几乎无变量的多元函数微积分的理论),
因为我的水平不够高.
我想, 也差不多了, 就打开\gls{vsc}, 用\gls{latex}写书.
上一次写代数书时过于随意, 没好好写序;
这一次, 我就想认真地写序.
自然地, 我想查一查前人是否有相关研究.
不出意外地, 查到了几篇资料.
我认真地看了看, 并修正了自己用的一些记号与理论.
可以说, 这是站在巨人的肩膀上的 ``读书报告'':
这本书 ``浪费了'' 巨人的肩膀, 并没有新鲜的\gls{math}.
不过, 我想, 最起码, 我还是能视这本书为算普读物的.

上一次, 我写代数书的时候, 我的\gls{latex}水平还比较低, 代码一团糟.
甚至, 最近, 我欲重编译它, 结果出现了错误
(我也不想管它了, 暂时就让它烂着吧).
这一次, 我写微积分读物, 内容简单一些, 代码也更规范一些了.
上一本书的一些 ``优良传统'' 也来到了这本书上:
开放源代码 (the Unlicense),
并给自己的书套用 CC0 许可协议, 让这本书进入公有领域.
当然, 如果您仅仅是读我的书, 对您而言,
这些 ``优良传统'' 是不重要的.

您可以去以下的二个网址的任意一个获取本书的最新版:
\begin{align*}
     & \text{\url{https://gitee.com/septsea/calculus-with-almost-no-variables}}  \\
     & \text{\url{https://github.com/septsea/calculus-with-almost-no-variables}}
\end{align*}

若您看到本书的错误, 我希望您毫不犹豫地告诉我;
我会修正错误的.

\begin{flushright}
    \theauthor\\
    2022~年 4~月 17~日
\end{flushright}
