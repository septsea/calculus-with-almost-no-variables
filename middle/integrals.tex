\chapter{积分}

本章讨论如何\emph{计算}积分;
这里的 ``积分'' 是 ``定积分'',
虽然我觉得 ``定'' 有些多馀.

我暂且用一会儿传统的记号,
告诉您我在本章会写什么东西吧.

具体地, 设 $f$ 是区间 $I$ 上的连续函数, 且 $a$, $b \in I$.
积分论告诉我们, $f$ 在 $[a, b]$ 上的积分
(或者, $f$ 在 $[b, a]$ 上的积分的相反数)
\begin{align*}
    \alpha = \int_{a}^{b} {f(x)\,\mathrm{d}x}
\end{align*}
存在.
所以, 我们可以专心地思考怎么算出结果
(而不必担心结果是否存在);
本章就告诉您一些计算积分的方法.

当然, 我还是会使用
\begin{align*}
    \int_{a}^{b} {f}
\end{align*}
表示 $\alpha$;
毕竟, 这是本书的一个大主题.

\section{不定积分与积分}

我们在上一章 ``乘热打铁地'' 证明了当时没用到的定理:

\NewtonLeibniz*

原则上, 我们可以用这个定理计算很多积分了.

\begin{example}
    因为 $\mathrm{cos}$ 的一个原函数是 $\mathrm{sin}$, 故
    \begin{align*}
        \int_{0}^{2\pi/4} {\mathrm{cos}} = \mathrm{sin} [2\pi/4] - \mathrm{sin} [0] = 1.
    \end{align*}
\end{example}

\begin{example}
    因为 $\mathrm{exp}$ 的一个原函数是 $\mathrm{exp}$, 故
    \begin{align*}
        \int_{0}^{1} {\mathrm{exp}} = \mathrm{exp} [1] - \mathrm{exp} [0] = \mathrm{e} - 1.
    \end{align*}
\end{example}

\begin{example}
    因为 $1/(1 + \iota^2)$ 的一个原函数是 $\mathrm{arctan}$, 故
    \begin{align*}
        \int_{-1}^{1} {\frac{1}{1 + \iota^2}} = \mathrm{arctan} [1] - \mathrm{arctan} [-1] = \frac{2\pi}{4}.
    \end{align*}
\end{example}

不过, 我们不妨先探索不定积分跟积分的联系.
在建立一定的联系后, 我们可以更有条理地算积分.

定理~\ref{thm:NewtonLeibniz} 是
Newton-Leibniz 公式的一个经典说法;
它描述了\emph{原函数}与\emph{积分}的关系.
自然地, 就有这样的问题:
有没有\emph{直接}描述%
\emph{不定积分}跟\emph{积分}的关系的说法呢?

姑且从传统的记号说起.
或许, 您还能想起来, 用传统的记号, 可写上述三个积分的计算为
\begin{align*}
    \int_{0}^{2\pi/4} {\cos {x}\,\mathrm{d}x}   & = {\sin {x}} \,\Bigg|_{0}^{2\pi/4} = 1,             \\
    \int_{0}^{1} {\exp {x}\,\mathrm{d}x}        & = {\exp {x}} \,\Bigg|_{0}^{1} = \mathrm{e} - 1,     \\
    \int_{-1}^{1} {\frac{\mathrm{d}x}{1 + x^2}} & = {\arctan {x}} \,\Bigg|_{-1}^{1} = \frac{2\pi}{4}.
\end{align*}
这里, ${\exp {x}}\, \bigg|_{0}^{1}$
就是 $\exp {1} - \exp {0}$
(或者, 按本书的记号, $\mathrm{exp}[1] - \mathrm{exp}[0]$)
的省略.
所以, 我们也可定义一个类似的记号.

\begin{definition}
    设 $f$ 是 $P \subset \mathbb{R}$ 上的函数.
    设 $a$, $b \in P$.
    定义
    \begin{align*}
        \big[ f \big]_{a}^{b} = f[b] - f[a].
    \end{align*}
    姑且称其为 ``bracket 运算%
    \index{bracketYun4Suan4@bracket 运算}''
    (\gls{chi}: 方括号运算).
\end{definition}

传统的记号在函数 $f(x)$
(这里, 为对照, 使用经典的函数记号)
的右侧画单条长竖线;
为清晰起见, 我用一对方括号包围函数 $f$.
毕竟, 形如
\begin{align*}
    1 + f(x)\,\Bigg|_{a}^{b}
\end{align*}
的文字是有歧义的:
这是 $1 + (f(b) - f(a))$
还是 $(1 + f(b)) - (1 + f(a))$ 呢?

\begin{theorem}
    设 $f$, $g$ 都是 $P \subset \mathbb{R}$ 上的函数.
    设 $a$, $b \in P$.
    Bracket 运算适合如下性质:
    \begin{itemize}
        \item 二个函数的和的 bracket
              等于二个函数的 bracket 的和, 即
              \begin{align*}
                  \big[ f + g \big]_{a}^{b} = \big[ f \big]_{a}^{b} + \big[ g \big]_{a}^{b}.
              \end{align*}
        \item 设 $k$ 为 $P$ 上的常函数.
              则
              \begin{align*}
                  \big[ k \big]_{a}^{b} = 0.
              \end{align*}
        \item 设 $k$ 为 $P$ 上的常函数.
              则
              \begin{align*}
                  \big[ kf \big]_{a}^{b} = k \cdot \big[ f \big]_{a}^{b}.
              \end{align*}
        \item 设 $Q \subset \mathbb{R}$ 上的函数 $h$ 的值域%
              是 $P$ 的子集.
              设 $c$, $d \in Q$.
              则
              \begin{align*}
                  \big[ f \circ h \big]_{c}^{d} = \big[ f \big]_{h[c]}^{h[d]}.
              \end{align*}
    \end{itemize}
\end{theorem}

\begin{proof}
    按定义论证这四条即可.

    $[ f + g ]_{a}^{b}$, 按定义, 就是 $(f + g)[b] - (f + g)[a]$.
    不过, 我们知道, $(f + g)[b] = f[b] + g[b]$;
    类似地, $(f + g)[a] = f[a] + g[a]$.
    所以, 二者的差就是
    \begin{align*}
        (f[b] + g[b]) - (f[a] + g[a])
        = (f[b] - f[a]) + (g[b] - g[a])
        = \big[ f \big]_{a}^{b} + \big[ g \big]_{a}^{b}.
    \end{align*}

    $[ k ]_{a}^{b}$, 按定义, 就是 $k[b] - k[a]$.
    可是, $k[b] = k[a] = k$, 故 $[ k ]_{a}^{b} = 0$.

    $[ kf ]_{a}^{b}$, 按定义, 就是 $(kf)[b] - (kf)[a]$.
    不过, $(kf)[b] = k[b] \cdot f[b] = k \cdot f[b]$;
    类似地, $(kf)[a] = k \cdot f[a]$.
    所以
    \begin{align*}
        k \cdot f[b] - k \cdot f[a] = k \cdot (f[b] - f[a]) = k \cdot \big[ f \big]_{a}^{b}.
    \end{align*}

    最后一个或许是最容易的:
    \begin{align*}
        \big[ f \circ h \big]_{c}^{d}
        = {} & (f \circ h)[d] - (f \circ h)[c]       \\
        = {} & f[h[d]] - f[h[c]]                     \\
        = {} & \big[ f \big]_{h[c]}^{h[d]}. \qedhere
    \end{align*}
\end{proof}

利用 bracket 运算, 我们可 ``换汤不换药''
地改写 Newton-Leibniz 公式:

\begin{theorem}[Newton-Leibniz]
    设 $I$ 为区间.
    设 $f$ 是 $I$ 上的连续函数.
    设 $F$ 是 $f$ 的原函数.
    则对任意 $a$, $b \in I$,
    \begin{align*}
        \int_{a}^{b} {f} = \big[ F \big]_{a}^{b}.
    \end{align*}
\end{theorem}

的确,
这个改写只是用 ``新鲜的'' bracket 运算包装了函数在二点的差.
不过, 这还只是一小步;
我马上就要迈一大步了.

我刚定义了函数的 bracket 运算;
那么函数集有没有 bracket 运算呢?

这问题, 其实是废话:
有就是有, 没有就是没有.
的确, 我刚才只是定义了函数的 bracket 运算,
而没有定义函数集的 bracket 运算.
不过这是大问题吗?
我现在就定义它.

\begin{definition}
    设 $A$ 是 $P \subset \mathbb{R}$ 上的函数集.
    设 $a$, $b \in P$. 定义
    \begin{align*}
        \big[ A \big]_{a}^{b}
        = {} & \bigg\{ \big[ f \big]_{a}^{b} \,\bigg|\, f \in A \bigg\} \\
        = {} & \{ f[b] - f[a] \mid f \in A \}.
    \end{align*}
\end{definition}

不难看出, bracket 运算变函数集为数集 (实数集的子集).
所以, 为研究函数集的 bracket 运算的性质, 我们要定义数集的运算.

\begin{definition}
    设 $P$, $Q$ 为 $\mathbb{R}$ 的子集.
    设 $\ast$ 是三文字 $+$, $-$, $\cdot$ 的任意一个.
    定义
    \begin{align*}
        P \ast Q = \{ p \ast q \mid \text{$p \in P$, $q \in Q$} \}.
    \end{align*}
    老样子, 可写 $P \cdot Q$ 为 $PQ$,
    写 $\{ 0 \} - P$ 为 $-P$.

    若 $P$ 的每一个元都不是零, 还可定义
    \begin{align*}
        \frac{Q}{P} = \Bigg\{ \frac{q}{p} \,\Bigg|\,  \text{$q \in Q$, $p \in P$} \Bigg\}.
    \end{align*}

    若对任意 $p \in P$, $q \in Q$, $p^q$ 有意义, 则还可定义
    \begin{align*}
        P^Q = \{ p^q \mid \text{$p \in P$, $q \in Q$} \}.
    \end{align*}
\end{definition}

不意外地, 我们有如下性质.

\begin{theorem}
    设 $P$, $Q$, $S$ 都是 $\mathbb{R}$ 的子集.
    则
    \begin{align*}
        \begin{array}{ll}
            P + Q = Q + P,             & PQ = QP,       \\
            (P + Q) + S = P + (Q + S), & (PQ)S = P(QS).
        \end{array}
    \end{align*}
\end{theorem}

% \begin{proof}
%     我不证了;
%     这跟函数集的相关性质太相似了.
%     还是老套路:
%     证明左边是右边的子集,
%     且右边是左边的子集.
%     您肯定得用到 $\mathbb{R}$ 的运算律.
% \end{proof}

由此, 我们就有如下的函数集的 bracket 运算律:

\begin{theorem}
    设 $A$, $B$ 都是 $P \subset \mathbb{R}$ 上的函数集.
    设 $a$, $b \in P$.
    \begin{itemize}
        \item 二个函数集的和的 bracket
              等于二个函数集的 bracket 的和, 即
              \begin{align*}
                  \big[ A + B \big]_{a}^{b} = \big[ A \big]_{a}^{b} + \big[ B \big]_{a}^{b}.
              \end{align*}
        \item 设函数集 $C$ 的每一个元都是 $P$ 上的常函数.
              则
              \begin{align*}
                  \big[ C \big]_{a}^{b} = \{ 0 \}.
              \end{align*}
        \item 设 $k$ 为 $P$ 上的常函数.
              则
              \begin{align*}
                  \big[ \{ k \} A \big]_{a}^{b} = \{ k \} \cdot \big[ A \big]_{a}^{b}.
              \end{align*}
        \item 设 $Q \subset \mathbb{R}$ 上的函数 $h$ 的值域%
              是 $P$ 的子集.
              设 $c$, $d \in Q$.
              则
              \begin{align*}
                  \big[ A \circ \{ h \} \big]_{c}^{d} = \big[ A \big]_{h[c]}^{h[d]}.
              \end{align*}
    \end{itemize}
\end{theorem}

\begin{proof}
    还是老套路: 相互包含.
    由于我已经建立了函数的 bracket 运算律,
    所以您的论证应该不会太长.
\end{proof}

在上一章, 我曾说, 在不引起混淆时,
可写恰含一个元的集 $\{ a \}$ 为 $a$.
现在我又要采用这个约定了.

\begin{theorem}
    设 $P \subset \mathbb{R}$.
    设 $f$ 是 $P$ 上的函数.
    设 $\mathbb{R}_P$ 是 $P$ 上的所有 (实的) 常函数作成的集
    (当然, 也可简单地写其为 $\mathbb{R}$).
    设 $a$, $b \in P$.
    则
    \begin{align*}
        \big[ f + \mathbb{R} \big]_{a}^{b} = \big[ f \big]_{a}^{b}.
    \end{align*}
\end{theorem}

\begin{proof}
    直接验证;
    不过, 您还是要注意一些细节的.
    \begin{align*}
        \big[ f + \mathbb{R} \big]_{a}^{b}
        = {} & \big[ \{f\} + \mathbb{R} \big]_{a}^{b}                     \\
        = {} & \big[ \{f\} \big]_{a}^{b} + \big[ \mathbb{R} \big]_{a}^{b} \\
        = {} & \big[ \{f\} \big]_{a}^{b} + \{ 0 \}                        \\
        = {} & \big[ \{f\} \big]_{a}^{b}                                  \\
        = {} & \big[ f \big]_{a}^{b}. \qedhere
    \end{align*}
\end{proof}

下面的命题更直接地焊接了不定积分与积分.

\index{NewtonLeibnizGong1Shi4@Newton-Leibniz 公式}

\begin{restatable}[Newton-Leibniz]{theorem}{NewNewtonLeibniz}
    设 $I$ 为区间.
    设 $f$ 是 $I$ 上的连续函数.
    则对任意 $a$, $b \in I$,
    \begin{align*}
        \int_{a}^{b} {f} = \Bigg[ \int {f} \Bigg]_{a}^{b}.
    \end{align*}
\end{restatable}

\begin{remark}
    严谨地 (但不重要地), 我们应当写
    \begin{align*}
        \Bigg\{ \int_{a}^{b} {f} \Bigg\} = \Bigg[ \int {f} \Bigg]_{a}^{b}.
    \end{align*}
\end{remark}

\begin{proof}
    因为 $f$ 是 $I$ 上的连续函数,
    故 $f$ 有一个原函数 $F$,
    且 $\int {f} = F + \mathbb{R}$.
    从而
    \begin{align*}
         & \Bigg[ \int {f} \Bigg]_{a}^{b}
        = \big[ F + \mathbb{R} \big]_{a}^{b}
        = \big[ F \big]_{a}^{b}
        = \int_{a}^{b} {f}. \qedhere
    \end{align*}
\end{proof}

\section{计算积分}

在上一节, 我们得到了新的 Newton-Leibniz 公式:

\NewNewtonLeibniz*

这将是本节的重点;
毕竟, 在大多数场合, \emph{具体}计算积分时, 还是要用它.

我们先从积分论\emph{借}三个公式.
它们比 Newton-Leibniz 公式更基础;
或者说, 在论证 Newton-Leibniz 公式时, 我们已经用到它们了.

\LinearityOfInegrals*

\AdditivityOfIntervalsOfIntegrals*

我们曾建立了不定积分的运算律;
现在, 我们试建立积分的更多的运算律.

先定义一个术语.

\begin{definition}
    设 $I$ 为区间.
    设 $f$ 是 $I$ 上的可导函数.
    若 $\mathrm{D}f$ 还是 $I$ 上的连续函数,
    则说 $f$ \emph{连导}\index{lian2dao3@连导}.
\end{definition}

\begin{theorem}
    设 $I$, $J$ 为\emph{区间}.
    设 $g$ 是 $J$ 上的\emph{连续函数}.
    设 $f$: $I \to J$ \emph{连导}.
    设 $a$, $b \in I$.
    则
    \begin{align*}
        \int_{a}^{b} {(g \circ f)\,\mathrm{D}f} = \int_{f[a]}^{f[b]} {g}.
    \end{align*}
\end{theorem}

请您注意上述定理的\emph{关键词};
为保证积分有意义
(请注意, 在本书, 我们只讨论\emph{连续函数}的积分论),
我要求函数的性质好一些;
不过, 这并不是很影响实际应用.

\begin{proof}
    因为 $f$ 连导, 故 $f$ 跟 $\mathrm{D}f$ 都是连续函数.
    因为 $g$ 是连续函数,
    故 $h = (g \circ f)\,\mathrm{D}f$ 也是连续函数.
    所以, 可用 Newton-Leibniz 公式计算 $h$ 的积分.
    于是
    \begin{align*}
        \int_{a}^{b} {(g \circ f)\,\mathrm{D}f}
        = {} & \Bigg[ \int {(g \circ f)\,\mathrm{D}f} \Bigg]_{a}^{b} \\
        = {} & \Bigg[ \Bigg( \int {g} \Bigg) \circ f \Bigg]_{a}^{b}  \\
        = {} & \Bigg[ \int {g} \Bigg]_{f[a]}^{f[b]}                  \\
        = {} & \int_{f[a]}^{f[b]} {g}. \qedhere
    \end{align*}
\end{proof}

我们无妨再看一遍命题的论证.
首先, 我们明确 $(g \circ f)\,\mathrm{D}f$ 是连续函数,
故可用 Newton-Leibniz 公式计算积分;
然后, 我们用不定积分的运算律转化为 $g$ 的积分
(因为 $g$ 是连续函数).

其实, 在实际的计算中,
我们\emph{并不需要}每一次都变不定积分的 bracket 运算为积分.
一般地, 我们用 Newton-Leibniz 公式算积分.
为方便, 我们一般代函数在二点的值的差
以长竖线 (当然, 在本书, 用 bracket 运算).
因为本书定义了函数集的 bracket 运算,
故我们可直接变积分为函数集的 bracket.
在方括号内, 就是我们已经熟知的不定积分.
所以, 不定积分与 bracket 运算就能解决大多数积分问题.
(当然, Newton-Leibniz 公式也不是全能的.)

\begin{example}
    我以简单的 $\mathrm{sin}\, \mathrm{cos}$ 为例吧.
    \begin{align*}
        \int_{0}^{2\pi/4} {\mathrm{sin}\,\mathrm{cos}}
        = {} & \int_{0}^{2\pi/4} {\mathrm{sin}\,\mathrm{D}\,\mathrm{sin}}
        = \int_{\sin[0]}^{\sin[2\pi/4]} {\iota}                                    \\
        = {} & \Bigg[ \int {\iota} \Bigg]_{\mathrm{sin}[0]}^{\mathrm{sin}[2\pi/4]}
        = \Bigg[ \frac{\iota^2}{2} \Bigg]_{0}^{1}                                  \\
        = {} & \frac{1}{2}.
    \end{align*}

    若充分利用 bracket 运算的性质, 我们可以这么写:
    \begin{align*}
        \int_{0}^{2\pi/4} {\mathrm{sin}\,\mathrm{cos}}
        = {} & \Bigg[ \int {\mathrm{sin}\,\mathrm{cos}} \Bigg]_{0}^{2\pi/4}             \\
        = {} & \Bigg[ \int {\mathrm{sin}\,\mathrm{D}\,\mathrm{sin}} \Bigg]_{0}^{2\pi/4}
        = \Bigg[ \Bigg( \int {\iota} \Bigg) \circ \mathrm{sin} \Bigg]_{0}^{2\pi/4}      \\
        = {} & \Bigg[ \int {\iota} \Bigg]_{\mathrm{sin}[0]}^{\mathrm{sin}[2\pi/4]}      \\
        = {} & \Bigg[ \frac{\iota^2}{2} + \mathbb{R} \Bigg]_{0}^{1}                     \\
        = {} & \Bigg[ \frac{\iota^2}{2} \Bigg]_{0}^{1}
        = \frac{1}{2}.
    \end{align*}
    当然, 因为 $\mathbb{R}$ 的 bracket 为 $0$,
    故计算积分时不必写出来.
\end{example}

\begin{theorem}
    设 $I$, $J$ 为\emph{区间}.
    设 $J$ 上的函数 $g$ \emph{连续}.
    设 $f$: $I \to J$ \emph{连导}.
    设 $e$: $J \to I$ 适合 $f \circ e = \iota$.
    任取 $c$, $d \in J$.
    则
    \begin{align*}
        \int_{c}^{d} {g} = \int_{e[c]}^{e[d]} {(g \circ f)\,\mathrm{D}f}.
    \end{align*}
\end{theorem}

\begin{proof}
    注意到 $g$ 与 $(g \circ f)\,\mathrm{D}f$ 都是连续函数.
    所以
    \begin{align*}
        \int_{c}^{d} {g}
        = {} & \Bigg[ \int {g} \Bigg]_{c}^{d}                          \\
        = {} & \Bigg[ \Bigg( \int {(g \circ f)\,\mathrm{D}f} \Bigg)
        \circ e \Bigg]_{c}^{d}                                         \\
        = {} & \Bigg[ \int {(g \circ f)\,\mathrm{D}f} \Bigg]
        _{e[c]}^{e[d]}                                                 \\
        = {} & \int_{e[c]}^{e[d]} {(g \circ f)\,\mathrm{D}f}. \qedhere
    \end{align*}
\end{proof}

\begin{example}
    计算
    \begin{align*}
        \int_{-1}^{1} {\mathrm{sqrt} \circ (1 - \iota^2)}.
    \end{align*}

    设 $J = [-1, 1]$.
    设 $g = \mathrm{sqrt} \circ (1 - \iota^2)$ 为 $J$ 上的函数.
    为计算此积分, 我们无妨考虑找一个合适的\emph{区间} $I$,
    与一个 $I$ 到 $J$ 的连导函数 $f$,
    使 $(g \circ f)\,\mathrm{D}f$ 简单一些;
    当然, 还要找一个 $J$ 到 $I$ 的函数 $e$,
    使 $f \circ e = \iota$.
    我们无妨取
    $I = [-2\pi/4, 2\pi/4]$, $f = \mathrm{sin}$.
    那么, $\mathrm{D}f = \mathrm{cos}$,
    且 $g \circ f = \mathrm{cos}$.
    并且, 不难看出, 取 $e = \mathrm{arcsin}$,
    即得 $f \circ e = \iota$.
    故
    \begin{align*}
        \int_{-1}^{1} {\mathrm{sqrt} \circ (1 - \iota^2)}
        = {} & \Bigg[ \int {\mathrm{sqrt} \circ (1 - \iota^2)} \Bigg]_{-1}^{1}   \\
        = {} & \Bigg[ \Bigg( \int {(\mathrm{sqrt} \circ (1 - \iota^2)
                \circ \mathrm{sin})\,\mathrm{D}\,\mathrm{sin}} \Bigg)
        \circ \mathrm{arcsin} \Bigg]_{-1}^{1}                                    \\
        = {} & \Bigg[ \int {\mathrm{cos}^2} \Bigg]
        _{\mathrm{arcsin}[-1]}^{\mathrm{arcsin}[1]}                              \\
        = {} & \Bigg[ \int {\frac{1 + \mathrm{cos}}{2} \circ {2\iota}} \Bigg]
        _{-2\pi/4}^{2\pi/4}                                                      \\
        = {} & \Bigg[ \Bigg( \int {\frac{1}{2} \cdot \frac{1 + \mathrm{cos}}{2}}
        \Bigg) \circ {2\iota} \Bigg]_{-2\pi/4}^{2\pi/4}                          \\
        = {} & \Bigg[ \int {\frac{1 + \mathrm{cos}}{4}}
        \Bigg]_{2\iota[-2\pi/4]}^{2\iota[2\pi/4]}                                \\
        = {} & \Bigg[ \frac{\iota + \mathrm{sin}}{4}
        \Bigg]_{-2\pi/2}^{2\pi/2}                                                \\
        = {} & \frac{2\pi/2 + 0}{4} - \frac{-2\pi/2 + 0}{4}                      \\
        = {} & \frac{2\pi}{4}.
    \end{align*}
\end{example}

\begin{theorem}
    设 $I$ 为\emph{区间},
    $f$, $g$ 都是 $I$ 上的\emph{连导}函数.
    设 $a$, $b \in I$.
    则
    \begin{align*}
        \int_{a}^{b} {f\mathrm{D}g} = \big[ fg \big]_{a}^{b} - \int_{a}^{b} {g\mathrm{D}f}.
    \end{align*}
\end{theorem}

\begin{proof}
    $f\mathrm{D}g$ 与 $g\mathrm{D}f$ 都是连续函数.
    从而
    \begin{align*}
        \int_{a}^{b} {f\mathrm{D}g}
        = {} & \Bigg[ \int {f\mathrm{D}g} \Bigg]_{a}^{b}
        = \Bigg[ fg - \int {g\mathrm{D}f} \Bigg]_{a}^{b}                          \\
        = {} & \big[ fg \big]_{a}^{b} - \Bigg[ \int {g\mathrm{D}f} \Bigg]_{a}^{b}
        = \big[ fg \big]_{a}^{b} - \int_{a}^{b} {f\mathrm{D}g}. \qedhere
    \end{align*}
\end{proof}

\begin{example}
    计算
    \begin{align*}
        \int_{0}^{2\pi/2} {\mathrm{sin}\, \mathrm{exp}}.
    \end{align*}

    注意到 $\mathrm{D}\,\mathrm{exp} = \mathrm{exp}$.
    所以
    \begin{align*}
        \int_{0}^{2\pi/2} {\mathrm{sin}\, \mathrm{exp}}
        = {} & \int_{0}^{2\pi/2} {\mathrm{sin}\, \mathrm{D}\,\mathrm{exp}}     \\
        = {} & \big[ \mathrm{sin}\,\mathrm{exp} \big]_{0}^{2\pi/2}
        - \int_{0}^{2\pi/2} {\mathrm{exp}\, \mathrm{D}\,\mathrm{sin}}          \\
        = {} & 0 - \int_{0}^{2\pi/2} {\mathrm{cos}\, \mathrm{exp}}             \\
        = {} & {-\int_{0}^{2\pi/2} {\mathrm{cos}\, \mathrm{D}\, \mathrm{exp}}} \\
        = {} & {-\big[ \mathrm{cos}\,\mathrm{exp} \big]}_{0}^{2\pi/2}
        + \int_{0}^{2\pi/2} {\mathrm{exp}\, \mathrm{D}\,\mathrm{cos}}          \\
        = {} & (1 + \mathrm{exp} [2\pi/2])
        - \int_{0}^{2\pi/2} {\mathrm{sin}\, \mathrm{exp}}.
    \end{align*}
    由此可知
    \begin{align*}
        \int_{0}^{2\pi/2} {\mathrm{sin}\, \mathrm{exp}} = \frac{1 + \mathrm{exp} [2\pi/2]}{2}.
    \end{align*}
\end{example}

最后, 我再讲一个小技巧吧.

\begin{theorem}
    设 $I$ 为区间.
    设 $f$ 是 $I$ 上的连续函数.
    设 $a$, $b \in I$.
    则
    \begin{align*}
        \int_{a}^{b} {f}
        = {} & \int_{a}^{b} {f \circ (a + b - \iota)}
        = \frac{1}{2} \int_{a}^{b} {(f + f \circ (a + b - \iota))}.
    \end{align*}
\end{theorem}

\begin{proof}
    我们证明前一个等号成立即可;
    至于后一个等号, 利用
    ``二个相等的数的平均数跟其相等''
    即可.
    \begin{align*}
        \int_{a}^{b} {f \circ (a + b - \iota)}
        = {} & \int_{a}^{b} {-f \circ (a + b - \iota) \cdot (-1)}                    \\
        = {} & \int_{a}^{b} {(-f \circ (a + b - \iota))\,\mathrm{D} [a + b - \iota]} \\
        = {} & \int_{(a + b - \iota)[a]}^{(a + b - \iota)[b]} {-f}                   \\
        = {} & {-\int_{b}^{a} {f}}
        = \int_{a}^{b} {f}. \qedhere
    \end{align*}
\end{proof}

\begin{example}
    计算
    \begin{align*}
        \int_{0}^{2\pi/2} {\frac{\iota \, \mathrm{sin}}{2 - \mathrm{sin}^2}}.
    \end{align*}

    设
    \begin{align*}
        f = \frac{\iota \, \mathrm{sin}}{2 - \mathrm{sin}^2}.
    \end{align*}
    按照以往的经验, 您可能会试着先找 $f$ 的不定积分.
    注意, $f$ 甚至可以是 $\mathbb{R}$ 上的连续函数,
    所以 $f$ 的不定积分\emph{一定存在}.
    只不过, 您是否能顺利地写出来, 就是另一个问题.
    我承认, 我自己也写不出初等函数 $F$, 使 $\mathrm{D}F = f$.

    既然找原函数不太行, 那就试着用小技巧.
    注意到
    \begin{align*}
        \mathrm{sin} = \mathrm{sin} \circ \bigg( \frac{2\pi}{2} - \iota \bigg),
    \end{align*}
    故
    \begin{align*}
        f \circ \bigg(0 + \frac{2\pi}{2} - \iota \bigg) = \frac{(2\pi/2 - \iota) \, \mathrm{sin}}{2 - \mathrm{sin}^2}.
    \end{align*}
    从而
    \begin{align*}
        g
        = {} & f + f \circ \bigg(0 + \frac{2\pi}{2} - \iota \bigg)          \\
        = {} & \frac{2\pi}{2} \cdot \frac{\mathrm{sin}}{2 - \mathrm{sin}^2}
        = \frac{2\pi}{2} \cdot \frac{\mathrm{sin}}{1 + (-\mathrm{cos})^2}.
    \end{align*}
    所以
    \begin{align*}
        \int_{0}^{2\pi/2} {f}
        = {} & \frac{1}{2} \cdot \int_{0}^{2\pi/2} {g}                          \\
        = {} & \frac{1}{2} \cdot \frac{2\pi}{2} \int_{0}^{2\pi/2}
        {\frac{\mathrm{D}[-\mathrm{cos}]}{1 + (-\mathrm{cos})^2}}               \\
        = {} & \frac{2\pi}{4} \cdot
        \int_{-\mathrm{cos}[0]}^{-\mathrm{cos}[2\pi/2]}
        {\frac{1}{1 + \iota^2}}                                                 \\
        = {} & \frac{2\pi}{4} \cdot \int_{-1}^{1} {\mathrm{D}\,\mathrm{arctan}} \\
        = {} & \frac{2\pi}{4} \cdot \frac{2\pi}{4}
        = \frac{(2\pi)^2}{16}.
    \end{align*}
\end{example}
