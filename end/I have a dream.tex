\chapter{I Have a Dream}

\begin{remark*}
    本章是用来测试排版的.
\end{remark*}

\begin{remark*}
    Martin Luther King the Younger gave his nameknown speech
    \textit{I Have a Dream} in Washingtonburg on the twenty-eighth
    of Weedmonth in 1963.
\end{remark*}

I am happy to gather with you today in what will go down as the
greatest throng for freedom in our homeland\textquotesingle{}s
tale. Five score years ago, a great American, in whose betokening
shadow we stand today, underwrote his name on the deed giving
freedom to the black thrall. This timely boding came as a great
beacon light of hope to black thralls in their micklereds who had
been seared in the wilms of withering wronghood. A frovering
daybreak it was to end the long night of thralldom.

But one hundred years later, the black man still is not free. One
hundred years later, the life of the black man is still sadly
crippled in the shackles of beshedding, One hundred years later,
the black man lives on a lonely iland of warmth in the midst of a
wide forblowing fivelway. One hundred years on, the black man is
still ailing in the herns of American fellowship, spurned in his
own land. And so we\textquotesingle{}ve come here today to unhele
a shameful fettle.

In a way we have come to our homeland\textquotesingle{}s headtown
to call in a draught. When our ledewealth\textquotesingle{}s
draughtsmen outspent the words of Lawwrit and Selfhood, they were
pledging a ledger to which every American was to fall erve from.
This was the begetting that all men, yes, black men as well as
white men, would be indowed the ``Shendless Rights'' of ``Life,
Freedom and the seeking of Happiness.'' Needless to edmind that
America has since lied on her oath, insofar as her black fellowmen
can reckon.

The deed was a hight that all men, yes, black and white would have
life\textquotesingle{}s yieldless rights, freedom and the right to
seek eadiness.

It is fair to see today that Americans have been found wanting in
fairness, doing little on this hightful deed in their dealings
with their black brothers. Rather than holding worthiness firmly
in their hearts in following up this hallowed call to right a
wrong, America has given its black folk a ungood draught. A
draught that has come back with the words ``not enough fee.''

But we unwilling to believe that the horden is without fairness or
fee. We also are unwilling to believe that there is not enough fee
in this land\textquotesingle{}s great hordern. So we have come to
take in fee this draught, a draught that will give upon asking
freedom\textquotesingle{}s boons and hele\textquotesingle{}s
fairness.

We have also come to this hallowed spot to bring to
America\textquotesingle{}s mind again Now\textquotesingle{}s
pressing need. This is not the time to take a cooling-off sop or
the calming healthdrug of let\textquotesingle{}s go forward
little-by-little.

Now is the time to make true this mighty hight.

Now it is the time for the black folk to rise from
aparthood\textquotesingle{}s darkness and lonely hollow into
fair-go\textquotesingle{}s sunlit path.

Now it is time to lift our homeland out of this folkstrandish
quicksand onto the rock of brotherly steadfastness.

Now is the time to give a fair deal to all God\textquotesingle{}s
children.

It would be dooming for the homeland to stay deaf to the black
folk\textquotesingle{}s thronging call for freedoms and rights and
underguess their steadfastness in seeking them now. Their
sweltering summer\textquotesingle{}s lawful gladlessness will not
go-away until there is freedom with fairness. Nineteen sixty-three
is not the end but a beginning. Those who hoped that the black
American needed only to let-off some steam and will now be
fulfilled will have a stark mindjarring awakening if the homeland
goes back to its old, unfair ways.

There will be neither be a frithsome soughing over America until
the black American is given his full rights. The uprising, like a
windwhirl, will shake our folkdom\textquotesingle{}s frame until
the sun shines fairly and evenly on all.

We can never be fulfilled as long as our bodies, weighed down and
tired with the day\textquotesingle{}s wayfaring cannot get board
and lodging in inns along our highways and in our great towns.

We cannot be fulfilled as long as the black folks leave small
wretchsteads to end-up only in larger wretchsteads.

We can never be fulfilled as long as our bairns have taken from
them their self-worth and have their selfhood reaved from them by
boards that read ``for whites only.''

We cannot be fulfilled as long as a black folk in Mississippi
cannot folk-aye and black folk in New York believe that they have
nothing for which to folk-aye.

No, no we are not fulfilled and we will not be fulfilled until
fairness flows on downwards like waters and righteousness fares
forth like a mighty stream.

I am not unmindful that many have come here today ordeal-wearied
and sorely smited. Others have come from steads where seeking your
freedoms has left you harried and hounded, and smitten by
harshness\textquotesingle{} biting winds, wrought upon you by
those given to uphold your rights and freedoms.

You have been old-hands at finding understanding and insight in
bearing the burden. Go on with your work with the belief that
dreeing an unearned weird will make you free.

Go back to Mississippi, go back to Alabama, go back to South
Carolina, go back to Georgia, go back to Louisiana, go back to the
wretchsteads, to the small black townships throughout our now
great towns, knowing that somehow this wrong can and will be made
right.

Let us not wallow in the yesterday\textquotesingle{}s waned and
withered hopes. I say to you, my friends, we have the burdens in
our heart and toils in our the mind, today and tomorrow.

I have a dream. It is a foresight deeply and longly rooted in the
American mind.

I have a dream that one day this folkdom will rise up and live out
the true meaning of its belief that all men are made even.

I have a dream that one day in Georgia\textquotesingle{}s red
hills one-time thralls\textquotesingle{} sons and one-time
thrall-owners\textquotesingle{} sons will sit down together at
brotherhood\textquotesingle{}s table.

I have a mindsight that one day Mississippi shire, a shire
sweltering under downtrodden-ness\textquotesingle{} heat, will be
shaped otherwisely into an lush well, brimming with freedom and
fairness.

I have a dream that my four little children will one day live in a
land where they will be deemed not by their hue, but by their
deeds.

I have a dream today.

I have a dream that one day down in Alabama, with
it\textquotesingle{}s evil-willed hindering haters, its leader
having his lips dripping with the words of ``getting in the way''
and ``overturning''; that one day right there in Alabama little
black children, carls and frows, can link hands with little white
carls and frows, as sisters and brothers.

I have a dream today.

I have a dream that every dale shall be swallowed-up, every hill
shall be lifted up and every berg shall be made low, the rough
places will be made smooth, and the crooked places will be made
straight and the Lord\textquotesingle{}s greatness shall be made
for all to see and all flesh shall see it together.

This is our hope. This is the belief that I will go back to the
South filled with. With this belief we will have the strength to
hew out from hopelessness\textquotesingle{} hill,
hope\textquotesingle{}s stone.

With this hope we can shape anew our heart clattering, sadly
beating for our land asundered, into a brotherhood gladdened and
gleeful.

With this belief we can work together, make our beseeching to God
together, to dree together, to be locked-up together, to climb up
for freedom together, knowing that we will be free one day.

This will be the day when all God\textquotesingle{}s bairns will
sing with new understanding ``My land \textquotesingle{}tis of
thee, sweet land of freedom, of thee I sing. Land where my fathers
died, land of the wayfarer\textquotesingle{}s pride, from every
fellside, let freedom ring!''

And if America is to be a great land, this must become true. So
let freedom ring from the hilltops in New Hampshire. And let
freedom from New York\textquotesingle{}s mighty bergs ring.

Let freedom ring from the heightening Alleghenies in Pennsylvania.

Let freedom ring from the snow-topped Rockies in Colorado.

Let freedom ring from California\textquotesingle{}s wendsome
slopes.

But not only that, let freedom ring from
Georgia\textquotesingle{}s Stony berg.

Let freedom ring from every hill and molehill throughout
Mississippi and along every bergside.

When we let freedom ring, when we let it ring from every
boarding-house and every small hamlet, from every shire and every
great town, we can speed up the day when all
God\textquotesingle{}s children, black and white, Jew and
un-Jews, Romish-church men and those who are not Romish churchmen, can link hands and sing the old song, in words sung by
this land\textquotesingle{}s enthralled black folk, Free at last,
free at last. ``Thank God Almighty, we are free at last.''
