\chapter{The Speeches of Professor Xenofon Zolotas}

\begin{remark*}
    本章是用来测试排版的.
\end{remark*}

\begin{remark*}
    In 1957 and 1959, the Greek economist Professor Xenofon
    Zolotas, Governor of the bank of Greece and Governor of the
    Funds for Greece, delivered two speeches in English using
    Greek words only. As Prof.~Zolotas said:

    ``I always wished to address this Assembly in Greek, but I
    realized that it would have been indeed Greek to all present
    in this room. I found out, however, that I could make my
    address in Greek which would still be English to everybody.
    With your permission, Mr.~Chairman, I shall do it now, using
    with the exception of articles and prepositions only Greek
    words.''
\end{remark*}

\section{The First Speech}

Kyrie,

I eulogize the archons of the Panethnic Numismatic Thesaurus and
the Ecumenical Trapeza for the orthodoxy of their axioms, methods
and policies, although there is an episode of cacophony of the
Trapeza with Hellas.

With enthusiasm we dialogue and synagonize at the synods of our
didymous Organizations in which polymorphous economic ideas and
dogmas are analyzed and synthesized.

Our critical problems such as the numismatic plethora generate
some agony and melancholy. This phenomenon is characteristic of
our epoch. But, to my thesis, we have the dynamism to program
therapeutic practices as a prophylaxis from chaos and catastrophe.

In parallel, a panethnic unhypocritical economic synergy and
harmonization in a democratic climate is basic.

I apologize for my eccentric monologue. I emphasize my eucharistia
to you Kyrie, to the eugenic and generous American Ethnos and to
the organizers and protagonists of this Amphictyony and the
gastronomic symposia.

\begin{flushright}
    September 26, 1957
\end{flushright}

\section{The Second Speech}

Kyrie,

It is Zeus\textquotesingle{} anathema on our epoch for the
dynamism of our economies and the heresy of our economic methods
and policies that we should agonise between the Scylla of
numismatic plethora and the Charybdis of economic anaemia.

It is not my idiosyncrasy to be ironic or sarcastic but my
diagnosis would be that politicians are rather cryptoplethorists.
Although they emphatically stigmatize numismatic plethora,
energize it through their tactics and practices.

Our policies have to be based more on economic and less on
political criteria.

Our gnomon has to be a metron between political, strategic and
philanthropic scopes. Political magic has always been
antieconomic.

In an epoch characterised by monopolies, oligopolies, menopsonies,
monopolistic antagonism and polymorphous inelasticities, our
policies have to be more orthological. But this should not be
metamorphosed into plethorophobia which is endemic among academic
economists.

Numismatic symmetry should not antagonize economic acme.

A greater harmonization between the practices of the economic and
numismatic archons is basic.

Parallel to this, we have to synchronize and harmonize more and
more our economic and numismatic policies panethnically.

These scopes are more practical now, when the prognostics of the
political and economic barometer are halcyonic.

The history of our didymous organisations in this sphere has been
didactic and their gnostic practices will always be a tonic to the
polyonymous and idiomorphous ethnical economics. The genesis of
the programmed organisations will dynamize these policies. I
sympathise, therefore, with the aposties and the hierarchy of our
organisations in their zeal to programme orthodox economic and
numismatic policies, although I have some logomachy with them.

I apologize for having tyrannized you with my hellenic
phraseology.

In my epilogue, I emphasize my eulogy to the philoxenous
autochthons of this cosmopolitan metropolis and my encomium to
you, Kyrie, and the stenographers.

\begin{flushright}
    October 2, 1959
\end{flushright}
