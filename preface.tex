\chapter*{前言}
\addcontentsline{toc}{chapter}{前言}
\markboth{前言}{前言}

\begin{definition*}
    $\text{算学} = \text{mathematics}$.
\end{definition*}

本书的标题是《几乎无变量的微积分》. 您按字面意思理解此标题就好; 本书讨论的微积分并不是没有变量, 而是减少了变量的使用. 本书的 ``微积分'' 跟 ``分析学'' 不一样. 我姑且这么描述: 分析学更偏向理论与理论间的联系 (如极限、连续、导数、积分等概念的联系), 而微积分更偏向具体的计算 (您当然可以认为 ``微积分'' 就是分析学; 这样的话, 本书的标题就应该是《几乎无变量地计算导数、不定积分、定积分》). 本书是一本算普读物 (算学普及读物). 本书并\emph{不是}从\emph{零}教您微积分 (假如我要写这样的书, 那我可能要更多时间与更多力气大改现有的微积分符号); 相反, 本书假定您会 (最基本的) 微积分. 这样, 我就可以专心展现无变量的微积分演算是什么样的. 您在看本书时, 可以拿我展现的计算过程跟微积分教材 (或高等算学教材, 也可以是算学分析教材) 作对比. 这样, 您可以看到这种 (几乎) 无变量的微积分在某些地方确实是有优势的.

本书的副标题是《\CalculusSubtitle 》. 其实, 您不必太在意这个副标题里的 ``\pageref{calculus:LastPage}~分''; 这只是一个 ``旧客'' (胡话: joke). 假如忽略本书的标题页、目录、前言、参考文献与参考文献前的页码为偶数的 (几乎) 空白页 (若存在), 那么本书就只有 \pageref{calculus:LastPage}~页. 我假定您至多用 1~分 (即 60~秒) 看 1~页. 这么看来, 您至多用 \pageref{calculus:LastPage}~分就可以了解最基本的微积分. 不过, 认真地, 若我至多用 1~分看 1~页\emph{算学书}, 我可能学不到什么东西.

我不是这本小书欲讨论的对象的创始人. 一位美籍奥地利裔算学家 Karl Menger 在 1949~年发表了名为 \textit{Are variables necessary in calculus?} 的文章. 一位捷克的数据科学家、语言学家、地理学家与音乐人 Jakub Marian 在 2014~年又提到了这个话题. 我在 2022~年~3 月也独立地搞出了一些东西. 不过, 我菜, 只搞出了 ``几乎无变量的一元微积分''. 当我想写这本小书时, 我才开始查阅文献. 不出意外, 我查到了一些资料 (不过并不是很多, 因为跟我的个人计算机焊接的互联网上的资源有限). 我仔细地阅读了这些资料, 并对自己的记号作出了一些改进. 我在参考文献里列出了无变量的微积分的文献, 您可以去看一看 (毕竟我不能很好地用文字表达我的想法).

相信大家都学过函数. 在初中算学里, 我们用变量定义函数. 下面是湘教版八年级下册的算学课本的定义.

\begin{definition*}
    在讨论的问题中, 称取值会发生变化的量为\emph{变量}, 称取值固定不变的量为\emph{常量} (或\emph{常数}).
\end{definition*}

\begin{definition*}
    一般地, 如果变量 $y$ 随着变量 $x$ 而变化, 并且对于 $x$ 取的每一个值, $y$ 都有唯一的一个值与它对应, 那么称 $y$ 是 $x$ 的\emph{函数}, 记作 $y = f(x)$. 这里的 $f(x)$ 是胡话 a function of $x$ (土话: $x$ 的函数) 的简记. 这时叫 $x$ 作\emph{自变量}, 叫 $y$ 作\emph{因变量}. 对于自变量 $x$ 取的每一个值 $a$, 称因变量 $y$ 的对应值为\emph{函数值}, 并记其作 $f(a)$.
\end{definition*}

这个定义, 虽不是很严谨, 但很形象. 至少, 刚接触 ``函数'' 的人会对函数有比较形象的认识. 早期的算学家就是用 ``这种函数'' 讨论微积分的. 不过, 随着算学的发展, 算学家需要对算学对象有严格的阐述. 函数也不例外. 1914~年, 德国算学家 Felix Hausdorff 在他的 \textit{Grundz{\"u}ge der Mengenlehre} 里用 ``有序对'' 定义函数 (在本书, 我也会这么定义函数). 这种定义当然避开了非算学话 ``变量'' ``对应''. 不过, 更严谨地看, ``有序对'' 是什么? 能不能用更基础的东西定义它? 1921~年, 波兰算学家 Kazimierz Kuratowski 在他的文章 \textit{Sur la notion de l{\textquotesingle}ordre dans la Th{\'e}orie des Ensembles} 里定义 $(a, b)$ 为 $\{ \{a\}, \{a, b\}\}$. 于是, 可以\emph{证明},
\begin{align*}
    (a, b) = (c, d) \iff \text{$a = c$ 且 $b = d$}.
\end{align*}
这样, Hausdorff 的定义就更完美了.

% 我并不打算在本书严谨地讨论集的理论, 所以我就提到这里.

尽管函数有现代的定义, 函数也有现代的、不带变量的记号 $f$ (而不是 $f(x)$), 可我们在进行微积分计算时, 还是用带变量的记号进行计算. 具体地, 我们计算导数时, 用的记号是
\begin{align*}
    f^{\prime} (x) \text{\quad 或 \quad} \frac{\mathrm{d}}{\mathrm{d}x} {f(x)};
\end{align*}
我们计算不定积分时, 用的记号是
\begin{align*}
    \int {f(x) \,\mathrm{d}x};
\end{align*}
我们计算定积分时, 用的记号是
\begin{align*}
    \int_{a}^{b} {f(x) \,\mathrm{d}x}.
\end{align*}

请允许我暂时跑题. 我并没有说这些记号不好. 相反, 这些记号十分经典, 经得起时间与算学家的考验. 我自己初学微积分 (与算学分析) 时, 就是用这套经典记号的. 比如, 可形象地写求导数的链规则为
\begin{align*}
    \frac{\mathrm{d}y}{\mathrm{d}x} = \frac{\mathrm{d}y}{\mathrm{d}u} \cdot \frac{\mathrm{d}u}{\mathrm{d}x},
\end{align*}
这里 $u = f(x)$, $y = g(u) = g(f(x))$. 视导数为 ``变率'', 那这就是在说, $y$ 关于 $x$ 的变率等于 $y$ 关于 $u$ 的变率与 $u$ 关于 $x$ 的变率的积. 很形象吧? 假设 A, B, C 三人在直线跑道上匀速前进. A 的速率是 B 的速率的 $\frac{11}{10}$ (也就是说, A 比 B 快 $\frac{1}{10}$), 而 B 的速率是 C 的速率的 $\frac{9}{10}$ (也就是说, B 比 C 慢 $\frac{1}{10}$), 那么 A 的速率是 C 的速率的 $\frac{99}{100}$; 这就是二个比的积.

回到正题. 我们已经看到, 我们通用的微积分记号带着朴素的函数思想. 此现象让我好奇. 我就想: ``有没有不要变量的微积分? 或者说, 有没有几乎不要变量的微积分?'' 我认真思考了几日. 至少, 我已经习惯用 $\mathrm{D}$ 表示求导, 所以导数似乎不是什么问题. 比方说, $\operatorname{D} \mathrm{exp} = \mathrm{exp}$, $\operatorname{D} \mathrm{cos} = -\mathrm{sin}$, $\operatorname{D} \mathrm{sin} = \mathrm{cos}$. 不过, 当我想表达 $\operatorname{D} \mathrm{ln}$ 时, 我意识到了一个重要的问题: ``已知 $\operatorname{D} \ln {x} = 1/x$. 左边的 $\ln {x}$ 就是 $\mathrm{ln}$, 可右边的 $1/x$ 应该是什么?'' 想起胡话里, reciprocal 是倒数的意思, 我就定义
\begin{align*}
    \text{$\mathrm{rec}$:} \quad
    \mathbb{R} \setminus \{ 0 \} & \to \mathbb{R} \setminus \{ 0 \}, \\
    x                            & \mapsto \frac{1}{x}.
\end{align*}
这样, 我就可以写 $\operatorname{D} \mathrm{ln} = \mathrm{rec}$. 不过, 我还是没法好好地表示 $\operatorname{D} \mathrm{arcsin}$ 跟 $\operatorname{D} \mathrm{arctan}$. 我这时才意识到, 因为在微积分里, 有名的 (是 named, 而不是 well-known 或 famous) 函数不够多, 所以我想表达普普通通的导数都要自己起名字. 不至于碰到一个函数就起名字吧? 所以, 我定义了所谓的 ``什么也不干'' 的函数
\begin{align*}
    \text{$\mathrm{fdn}$:} \quad
    \mathbb{R} & \to \mathbb{R}, \\
    x          & \mapsto x
\end{align*}
($\mathrm{fdn}$ 乃 the function that does nothing 之略). 这样, 再利用函数的运算, 我总算能无变量地写出基本的求导公式了.

我随后又作出了无变量不定积分与无变量定积分的理论. 不过, 我写不下去了 (没作出几乎无变量的多元函数微积分的理论), 因为我的水平不够高. 我想, 也差不多了, 就打开视觉工作室代码, 用乳胶写书. 上一次写代数书时过于随意, 没好好写前言; 这一次, 我就想认真地写前言. 自然地, 我想查一查前人是否有相关研究. 不出意外地, 查到了几篇资料. 我认真地看了看, 并修正了自己用的一些记号与理论. 可以说, 这是站在巨人的肩膀上的 ``读书报告'': 这本书 ``浪费了'' 巨人的肩膀, 并没有新鲜的算学. 不过, 我想, 最起码, 我还是能视这本书为算普读物的.

上一次, 我写代数书的时候, 我的乳胶水平还比较低, 代码一团糟. 甚至, 最近, 我欲重编译它, 结果出现了错误 (我也不想管它了, 暂时就让它烂着吧). 这一次, 我写微积分读物, 内容简单一些, 代码也更规范一些了. 上一本书的一些 ``优良传统'' 也来到了这本书上: 开源代码 (the Unlicense), 并给自己的书套用 CC0 许可协议, 让这本书进入公有领域. 当然, 如果您仅仅是读我的书, 对您而言, 这些 ``优良传统'' 是不重要的.

您可以去以下的二个网址的任意一个获取本书的最新版:
\begin{align*}
     & \text{\url{https://gitee.com/septsea/calculus-with-almost-no-variables}}  \\
     & \text{\url{https://github.com/septsea/calculus-with-almost-no-variables}}
\end{align*}

最后, 我向一位取不来名字的网友表示感谢.

\begin{flushright}
    纳纳米\\
    \today
\end{flushright}
