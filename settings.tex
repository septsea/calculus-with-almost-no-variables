\documentclass[UTF8,fontset=none,zihao=-4,a4paper,leqno]{ctexbook}
\ctexset{today=big}

\usepackage{amssymb}
\usepackage{amsmath}
\usepackage{amsthm}
\usepackage{mathtools}
\usepackage{thmtools}
\usepackage{thm-restate}
\usepackage[hidelinks,hyperindex]{hyperref}
% Add hyperlinks to the table of contents
\usepackage{bookmark}
% partial=upright makes \partial stand up
\usepackage[math-style=ISO,bold-style=ISO,mathrm=sym,mathbf=sym,partial=upright,warnings-off={mathtools-colon,mathtools-overbracket}]{unicode-math}

\usepackage{fancyhdr}
\pagestyle{fancy}
\fancyhf{}
\setlength{\headheight}{15pt}
\fancyhead[LO]{\sffamily\nouppercase{\rightmark}}
\fancyhead[RO]{\sffamily\thepage}
\fancyhead[RE]{\sffamily\nouppercase{\leftmark}}
\fancyhead[LE]{\sffamily\thepage}
\renewcommand{\headrulewidth}{0pt}
\renewcommand{\footrulewidth}{0pt}
\fancypagestyle{plain}{%
    \fancyhf{}
    \fancyfoot[C]{\sffamily\thepage}
    \renewcommand{\headrulewidth}{0pt}
    \renewcommand{\footrulewidth}{0pt}
}

\allowdisplaybreaks[3]
\AtBeginDocument{
    % Standing-up-straight pi
    \let\umathpi\pi
    \renewcommand\pi{\symup\umathpi}
    \let\umathiota\iota
    \renewcommand\iota{\symup\umathiota}

    \renewcommand\leq{\leqslant}
    \renewcommand\geq{\geqslant}
    \renewcommand\mathellipsis{\cdots}
}

\usepackage{enumitem}
\setlist[1]{labelindent=\parindent}
\setlist[itemize]{leftmargin=*,nosep}
\setlist[enumerate]{leftmargin=*,nosep}
\setlist[enumerate,1]{label=(\arabic*),ref=(\arabic*)}
\setlist[description]{style=nextline,leftmargin=2\parindent}

\makeatletter
\providecommand{\openrighton}{}
\renewcommand{\openrighton}{\@openrighttrue}
\providecommand{\openrightoff}{}
\renewcommand{\openrightoff}{\@openrightfalse}
\makeatother

\declaretheoremstyle[
    spaceabove=2ex,
    spacebelow=2ex,
    headfont=\normalfont\bfseries,
    notefont=\mdseries,
    notebraces={(}{)},
    bodyfont=\normalfont,
    headindent=\parindent,
    headpunct={},
    postheadspace=1em
]{myStyle}

\declaretheoremstyle[
    spaceabove=2ex,
    spacebelow=2ex,
    headfont=\normalfont\bfseries,
    notefont=\mdseries,
    notebraces={(}{)},
    bodyfont=\normalfont,
    headindent=\parindent,
    headpunct={},
    postheadspace=1em,
    qed=\qedsymbol
]{proofStyle}

\declaretheorem[style=myStyle,name=定理,numberwithin=chapter]{theorem}
\declaretheorem[style=myStyle,name=定义,sibling=theorem]{definition}
\declaretheorem[style=myStyle,name=定义,numbered=no]{definition*}
\declaretheorem[style=myStyle,name=命题,sibling=theorem]{proposition}
\declaretheorem[style=myStyle,name=引理,sibling=theorem]{lemma}
\declaretheorem[style=myStyle,name=公理,sibling=theorem]{axiom}
\declaretheorem[style=myStyle,name=推论,sibling=theorem]{corollary}
\declaretheorem[style=myStyle,name=例,sibling=theorem]{example}
\declaretheorem[style=myStyle,name=注,sibling=theorem]{remark}
\declaretheorem[style=myStyle,name=注,numbered=no]{remark*}
\declaretheorem[style=myStyle,name=猜想,sibling=theorem]{conjecture}

% Undefine the proof environment provided by `amsthm`
\makeatletter
\let\proof\@undefined
\let\endproof\@undefined
\makeatother
\declaretheorem[style=proofStyle,name=证,numbered=no]{proof}

\let\emph\relax
\DeclareTextFontCommand{\emph}{\bfseries}

% Punctuation style: narrow some punctuation symbols
\punctstyle{kaiming}

\usepackage[sort&compress]{gbt7714}
\bibliographystyle{gbt7714-numerical}

\setmainfont{XITS}
% Standing-up-straight integral symbol (For XITS Math)
\setmathfont{XITS Math}[StylisticSet=8]
\setmathfont{TeX Gyre Termes Math}[range=bb/{latin,Latin}]
\setmathfont{STIX Two Math}[range={"2218-"2218}]
\setsansfont{Fira Sans Light}[BoldFont=Fira Sans SemiBold]
\setmonofont{Sarasa Mono SC Light}

\setCJKmainfont[ItalicFont=Sarasa Mono SC Light,BoldFont=Sarasa Mono SC Semibold]{Source Han Serif CN}
\setCJKsansfont{Sarasa Mono SC Light}
\setCJKmonofont{Sarasa Mono SC Light}

% \usepackage{makeidx}
% \makeindex

\def\qedsymbol{证毕.}

\renewcommand\thmcontinues[1]{%
\ifcsname hyperref\endcsname
    \hyperref[#1]{续页}
\else
    续页
\fi
{}\pageref{#1}%
}
