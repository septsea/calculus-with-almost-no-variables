\chapter{导数}

本章简单地提及导数及其运算.

若无特别说明, 本章的函数的定义域都是区间.

\section{背景}

\begin{definition}
    设 $I$ 为区间. 设 $f$ 为 $I$ 上的函数. 设 $x \in I$. 若存在 $x$ 的邻域 $N$, 与 $N \cap I$ 上的函数 $F$, 使
    \begin{align*}
        f = f[x] + (\iota - x)F,
    \end{align*}
    且 $F$ 于 $x$ 连续, 则说 $f$ 于 $x$ \emph{可导}, 并称 $F[x]$ 为 $f$ 于 $x$ 的\emph{导数}.
\end{definition}

\begin{remark}
    传统地, 我们用极限
    \begin{align*}
        \lim_{t \to x} {\frac{f[t] - f[x]}{t - x}}
    \end{align*}
    是否存在定义 $f$ 是否于 $x$ 可导; 极限存在时, 它的值就是 $f$ 于 $x$ 的导数. 可以证明, 这二个定义是等价的; 不过, 既然我花了不少篇幅讨论连续函数, 我将呈现一种不一样的微分学 (求导学). 我采取的定义来自希腊算学家 Constantin Carath{\'e}odory. 假如您对此事感兴趣, 您可以阅读美国算学家 Stephen Kuhn 的名为 \textit{The Derivative {\'a} la Carath{\'e}odory} 的文章 (不过, 我想说, 标题的 \textit{{\'a} la} 应该是 \textit{{\`a} la}).
\end{remark}

我们看导数的一些基本性质. 在定义里, 若 $f$ 于 $x$ 可导, 则 $f$ 于 $x$ 的导数似乎不止一个. 不过, 我们即将说明, 导数是唯一的.

\begin{theorem}
    设 $I$ 为区间. 设 $f$ 为 $I$ 上的函数. 设 $x \in I$. 设存在 $x$ 的邻域 $N_1$, 与 $N_1 \cap I$ 上的函数 $F_1$, 使
    \begin{align*}
        f = f[x] + (\iota - x) F_1,
    \end{align*}
    且 $F_1$ 于 $x$ 连续. 设存在 $x$ 的邻域 $N_2$, 与 $N_2 \cap I$ 上的函数 $F_2$, 使
    \begin{align*}
        f = f[x] + (\iota - x) F_2,
    \end{align*}
    且 $F_2$ 于 $x$ 连续. 则 $F_1 [x] = F_2 [x]$. 也就是说, $f$ 于 $x$ 的导数, 若存在, 则唯一.
\end{theorem}

\begin{proof}
    用反证法. 设 $F_1 [x] \neq F_2 [x]$, 则 $\varepsilon = |F_1 [x] - F_2 [x]|$ 是正数. 我们要由此推出矛盾.

    因为 $N_1 \cap I$ 是区间, 且 $F_1$ 于 $x$ 连续, 故存在正数 $\delta_1$, 使 $0 < |t - x| < \delta_1$ 且 $t \in I$ 时, 必有 $|F_1 [t] - F_1 [x]| < {\varepsilon}/{2}$, 即
    \begin{align*}
        \Bigg| \frac{f[t] - f[x]}{t - x} - F_1 [x] \Bigg| < \frac{\varepsilon}{2}.
    \end{align*}
    因为 $N_2 \cap I$ 是区间, 且 $F_2$ 于 $x$ 连续, 故存在正数 $\delta_2$, 使 $0 < |t - x| < \delta_2$ 且 $t \in I$ 时, 必有 $|F_2 [t] - F_2 [x]| < {\varepsilon}/{2}$, 即
    \begin{align*}
        \Bigg| \frac{f[t] - f[x]}{t - x} - F_2 [x] \Bigg| < \frac{\varepsilon}{2}.
    \end{align*}
    取 $\delta$ 为 $\delta_1$ 与 $\delta_2$ 中的较小者. 则 $0 < |t - x| < \delta$ 且 $t \in I$ 时,
    \begin{align*}
        \varepsilon
        = {}    & |F_1 [x] - F_2 [x]|                                                                                                 \\
        = {}    & \Bigg| \Bigg( \frac{f[t] - f[x]}{t - x} - F_2 [x] \Bigg) - \Bigg( \frac{f[t] - f[x]}{t - x} - F_1 [x] \Bigg) \Bigg| \\
        \leq {} & \Bigg| \frac{f[t] - f[x]}{t - x} - F_2 [x] \Bigg| + \Bigg| \frac{f[t] - f[x]}{t - x} - F_1 [x] \Bigg|               \\
        < {}    & \frac{\varepsilon}{2} + \frac{\varepsilon}{2} = \varepsilon.
    \end{align*}
    这是矛盾.
\end{proof}

\begin{example}
    设 $a$, $b$ 为实数. 则 $a\iota + b$ 于其定义域的任意一点都可导. 具体地,
    \begin{align*}
        a\iota + b = (a\iota + b)[x] + (\iota - x)a,
    \end{align*}
    而 $a$ 是连续函数. 并且, $a\iota + b$ 于任意一点的导数都是 $a$.
\end{example}

\begin{theorem}
    设 $I$ 为区间. 设 $f$ 为 $I$ 上的函数. 设 $x \in I$. 设 $f$ 于 $x$ 可导. 则 $f$ 于 $x$ 连续.
\end{theorem}

\begin{proof}
    因为 $f$ 于 $x$ 可导, 故存在 $x$ 的邻域 $N$, 与 $N \cap I$ 上的函数 $F$, 使
    \begin{align*}
        f = f[x] + (\iota - x)F,
    \end{align*}
    且 $F$ 于 $x$ 连续.

    因为 $\iota - x$ 于 $x$ 连续, 故 $(\iota - x)F$ 于 $x$ 连续; 因为 $f[x]$ 于 $x$ 连续, 故 $f$ 于 $x$ 连续.
\end{proof}

\begin{theorem}
    设 $I$ 为区间. 设 $f$, $g$ 为 $I$ 上的函数. 设 $x \in I$. 设 $f$, $g$ 都于 $x$ 可导. 则:
    \begin{itemize}
        \item $f + g$ 于 $x$ 可导, 且
              \begin{align*}
                  \text{($f + g$ 于 $x$ 的导数)} = \text{($f$ 于 $x$ 的导数)} + \text{($g$ 于 $x$ 的导数)}.
              \end{align*}
        \item 设 $k$ 为常数. 则 $kf$ 于 $x$ 可导, 且
              \begin{align*}
                  \text{($kf$ 于 $x$ 的导数)} = k \cdot \text{($f$ 于 $x$ 的导数)}.
              \end{align*}
        \item $fg$ 于 $x$ 可导, 且
              \begin{align*}
                  \text{($fg$ 于 $x$ 的导数)} = \text{($f$ 于 $x$ 的导数)} \cdot g[x] + f[x] \cdot \text{($g$ 于 $x$ 的导数)}.
              \end{align*}
        \item 若 $f[x] \neq 0$, 则 $g/f$ 于 $x$ 可导, 且
              \begin{align*}
                  \text{(${g}/{f}$ 于 $x$ 的导数)} = \frac{\text{($g$ 于 $x$ 的导数)} \cdot f[x] - g[x] \cdot \text{($f$ 于 $x$ 的导数)}}{f^2 [x]}.
              \end{align*}
    \end{itemize}
\end{theorem}

\begin{proof}
    因为 $f$ 于 $x$ 可导, 故存在 $x$ 的邻域 $N_1$, 与 $N_1 \cap I$ 上的函数 $F$, 使
    \begin{align*}
        f = f[x] + (\iota - x)F,
    \end{align*}
    且 $F$ 于 $x$ 连续. 类似地, 因为 $g$ 于 $x$ 可导, 故存在 $x$ 的邻域 $N_2$, 与 $N_2 \cap I$ 上的函数 $G$, 使
    \begin{align*}
        g = g[x] + (\iota - x)G,
    \end{align*}
    且 $G$ 于 $x$ 连续. 取 $N = N_1 \cap N_2$, 则 $N$ 也是 $x$ 的邻域. 这里, 为方便, 无妨滥用记号, 视 $F$, $G$ 分别是 $F$, $G$ 在 $N \cap I$ 上的限制. 此外, 注意到, $f$, $g$ 于 $x$ 的导数分别为 $F[x]$, $G[x]$.

    因为
    \begin{align*}
        f + g
        = {} & (f[x] + (\iota - x)F) + (g[x] + (\iota - x)G) \\
        = {} & (f[x] + g[x]) + (\iota - x)(F + G)            \\
        = {} & (f + g)[x] + (\iota - x)(F + G),
    \end{align*}
    且 $F + G$ 于 $x$ 连续, 故 $f + g$ 于 $x$ 可导, 且导数为 $(F + G)[x] = F[x] + G[x]$.

    因为
    \begin{align*}
        kf
        = {} & k(f[x] + (\iota - x)F)     \\
        = {} & kf[x] + k(\iota - x)F      \\
        = {} & (kf)[x] + (\iota - x)(kF),
    \end{align*}
    且 $kF$ 于 $x$ 连续, 故 $kf$ 于 $x$ 可导, 且导数为 $(kF)[x] = k \cdot F[x]$.

    因为
    \begin{align*}
        fg
        = {} & (f[x] + (\iota - x)F)(g[x] + (\iota - x)G)                                \\
        = {} & f[x]g[x] + f[x](\iota - x)G + (\iota - x)Fg[x] + (\iota - x)F(\iota - x)G \\
        = {} & (fg)[x] + (F \cdot g[x] + f[x] \cdot G + FG(\iota - x))(\iota - x),
    \end{align*}
    且 $h = F \cdot g[x] + f[x] \cdot G + FG(\iota - x)$ 于 $x$ 连续, 故 $fg$ 于 $x$ 可导, 且导数为 $h[x] = F[x]g[x] + f[x]G[x]$.

    最后一个等式需要一点儿技巧. 首先, 既然 $f[x] \neq 0$, 且 $f$ 于 $x$ 连续, 故存在 $x$ 的邻域 $M$, 使 $t \in M \cap I$ 时,
    \begin{align*}
        |f[t] - f[x]| < \frac{|f[x]|}{2},
    \end{align*}
    从而
    \begin{align*}
        |f[x]| = |f[t] - (f[t] - f[x])| \leq |f[t]| + |f[t] - f[x]| < |f[t]| + \frac{|f[x]|}{2},
    \end{align*}
    也就是
    \begin{align*}
        |f[t]| > |f[x]| - \frac{|f[x]|}{2} = \frac{|f[x]|}{2}.
    \end{align*}
    所以, 在 $M$ 上, $f$ 不取零值. 从而, 在 $(M \cap N) \cap I$ 上,
    \begin{align*}
        f^{-1}
        = {} & f^{-1} [x] + \Bigg( \frac{1}{f} - \frac{1}{f[x]} \Bigg)           \\
        = {} & f^{-1} [x] - \frac{f - f[x]}{f[x] f}                              \\
        = {} & f^{-1} [x] - \frac{(\iota - x)F}{f[x] f}                          \\
        = {} & f^{-1} [x] + (\iota - x) \cdot \Bigg( {-\frac{F}{f[x] f}} \Bigg).
    \end{align*}
    因为 $-F/(f[x]f)$ 于 $x$ 连续, 故 $f^{-1}$ 于 $x$ 可导, 且导数为 $-F[x]/(f^2 [x])$.

    注意到 $g/f = g \cdot f^{-1}$, 故 $g/f$ 于 $x$ 可导, 且
    \begin{align*}
             & \text{(${g}/{f}$ 于 $x$ 的导数)}                                                                                      \\
        = {} & \text{($gf^{-1}$ 于 $x$ 的导数)}                                                                                      \\
        = {} & \text{($g$ 于 $x$ 的导数)} \cdot f^{-1} [x] + g[x] \cdot \text{($f^{-1}$ 于 $x$ 的导数)}                                  \\
        = {} & \text{($g$ 于 $x$ 的导数)} \cdot \frac{1}{f[x]} + g[x] \cdot \Bigg({-\frac{\text{($f$ 于 $x$ 的导数)}}{f[x] f[x]}} \Bigg) \\
        = {} & \frac{\text{($g$ 于 $x$ 的导数)} \cdot f[x] - g[x] \cdot \text{($f$ 于 $x$ 的导数)}}{f^2 [x]}. \qedhere
    \end{align*}
\end{proof}

\begin{example}
    我们知道, $a\iota + b$: $\mathbb{R} \to \mathbb{R}$ 于 $\mathbb{R}$ 的任意一点 $x$ 都可导, 且导数为 $a$. 特别地, 常函数于 $x$ 的导数为 $0$, 而 $\iota$ 于 $x$ 的导数为 $1$. 利用算学归纳法, 可算出, 当 $n$ 为正整数时, $\iota^n$: $\mathbb{R} \to \mathbb{R}$ 于 $\mathbb{R}$ 的任意一点 $x$ 都可导, 且导数为 $nx^{n-1}$. 由此, 多项式函数于 $\mathbb{R}$ 的任意一点 $x$ 都可导. 进而, 有理函数于有定义的点可导. 顺便一提, 即使 $n$ 为负整数, $\iota^n$ 于非零的 $x$ 的导数仍为 $nx^{n-1}$.
\end{example}

\begin{theorem}[链规则]
    设 $I$, $J$ 为区间. 设 $f$: $I \to J$, $g$: $J \to K$. 设 $x \in I$. 设 $f$ 于 $x$ 可导, 且 $g$ 于 $f[x]$ 可导. 则 $g \circ f$ 于 $x$ 可导, 且
    \begin{align*}
        \text{($g \circ f$ 于 $x$ 的导数)} = \text{($g$ 于 $f[x]$ 的导数)} \cdot \text{($f$ 于 $x$ 的导数)}.
    \end{align*}
\end{theorem}

\begin{proof}
    因为 $g$ 于 $f[x]$ 可导, 故存在 $f[x]$ 的邻域 $M$, 与 $M \cap J$ 上的函数 $G$, 使
    \begin{align*}
        g = g[f[x]] + (\iota - f[x])G,
    \end{align*}
    且 $G$ 于 $f[x]$ 连续; 同时, $G[f[x]]$ 即为 $g$ 于 $f[x]$ 的导数. 因为 $f$ 于 $x$ 可导, 故存在 $x$ 的邻域 $N_1$, 与 $N_1 \cap I$ 上的函数 $F$, 使
    \begin{align*}
        f = f[x] + (\iota - x)F,
    \end{align*}
    且 $F$ 于 $x$ 连续; 同时, $F[x]$ 即为 $f$ 于 $x$ 的导数.

    因为 $f$ 亦于 $x$ 连续, 故对 $f[x]$ 的邻域 $M$ 来说, 必存在 $x$ 的邻域 $N_2$, 使 $f[N_2 \cap I] \subset M$; 又因为 $f$ 的陪域为 $J$, 故 $f[N_2 \cap I] \subset J$. 所以, $f[N_2 \cap I] \subset M \cap J$. 取 $N = N_1 \cap N_2$, 则 $N$ 也是 $x$ 的邻域, 且 $f[N \cap I] \subset M \cap J$. 为方便, 无妨滥用记号, 视 $F$, $f$ 分别是 $F$, $f$ 在 $N \cap I$ 上的限制. (事实上, 本段文字只是保证复合 $G \circ f$ 有意义罢了.)

    现在考察 $g \circ f$:
    \begin{align*}
        g \circ f
        = {} & g[f[x]] + (f - f[x])(G \circ f)                  \\
        = {} & (g \circ f)[x] + (\iota - x)F(G \circ f)         \\
        = {} & (g \circ f)[x] + (\iota - x) \cdot (G \circ f)F.
    \end{align*}
    因为 $G$ 于 $f[x]$ 连续, 而 $f$ 于 $x$ 连续, 故 $G \circ f$ 于 $x$ 连续; 又因 $F$ 于 $x$ 连续, 故 $(G \circ f)F$ 于 $x$ 连续. 所以 $g \circ f$ 于 $x$ 可导, 且导数为 $G[f[x]] \cdot F[x]$.
\end{proof}

\begin{theorem}
    设 $I$, $J$ 为区间. 设 $f$: $I \to J$ 严单调, 且有反函数 $f^{[-1]}$: $J \to I$. 设 $y \in J$. 设 $f$ 于 $f^{[-1]} [y]$ 可导. 则 $f^{[-1]}$ 于 $y$ 可导的一个必要与充分条件是: $f$ 于 $f^{[-1]} [y]$ 的导数非零.

    若 $f$ 于 $f^{[-1]} [y]$ 的导数不等于零, 则
    \begin{align*}
        \text{($f^{[-1]}$ 于 $y$ 的导数)} = \frac{1}{\text{($f$ 于 $f^{[-1]} [y]$ 的导数)}}.
    \end{align*}
\end{theorem}

\begin{proof}
    先看必要性. 既然 $f^{[-1]}$ 于 $y$ 可导, 且 $f$ 于 $f^{[-1]} [y]$ 可导, 那么 $\iota_J = f \circ f^{[-1]}$ 于 $y$ 可导. 由链规则, 有
    \begin{align*}
        1 = \text{($f$ 于 $f^{[-1]} [y]$ 的导数)} \cdot \text{($f^{[-1]}$ 于 $y$ 的导数)}.
    \end{align*}
    从而 $f$ 于 $f^{[-1]} [y]$ 的导数非零.

    再看充分性. 因为 $f$ 于 $f^{[-1]} [y]$ 可导, 故存在 $f^{[-1]} [y]$ 的邻域 $N_1$, 与 $N_1 \cap I$ 上的函数 $F$, 使
    \begin{align*}
        f = f[f^{[-1]} [y]] + (\iota - f^{[-1]} [y])F,
    \end{align*}
    且 $F$ 于 $f^{[-1]} [y]$ 连续. 依假定, $f$ 于 $f^{[-1]} [y]$ 的导数 $F[f^{[-1]} [y]] \neq 0$. 所以, 存在 $f^{[-1]} [y]$ 的邻域 $N_2$, 使在 $N_2 \cap (N_1 \cap I)$ 上, $F$ 不取零值. 取 $N = N_2 \cap N_1$, 则 $N$ 也是 $f^{[-1]} [y]$ 的邻域, 且在 $N \cap I$ 上, 不但 $F$ 不取零值, 且
    \begin{align*}
        f = y + (\iota - f^{[-1]} [y])F.
    \end{align*}

    因为 $f$ 严单调, 且 $I$, $J$ 都是区间, 故 $f$, $f^{[-1]}$ 都是连续函数. 特别地, $f^{[-1]}$ 于 $y$ 连续. 故对 $f^{[-1]} [y]$ 的邻域 $N$, 存在 $y$ 的邻域 $M$, 使 $f^{[-1]} [M \cap J] \subset N$. 因为 $f^{[-1]}$ 的值域为 $I$, 故 $f^{[-1]} [M \cap J] \subset I$. 也就是说, $f^{[-1]} [M \cap J] \subset N \cap I$. (事实上, 这只是为保证复合 $F \circ f^{[-1]}$ 有意义.) 那么, 在 $M \cap J$ 上, 有
    \begin{align*}
        f \circ f^{[-1]} = y + (f^{[-1]} - f^{[-1]} [y]) (F \circ f^{[-1]}),
    \end{align*}
    即
    \begin{align*}
        f^{[-1]} = f^{[-1]} [y] + (\iota - y) \cdot \frac{1}{F \circ f^{[-1]}}.
    \end{align*}
    因为 $f^{[-1]}$ 于 $y$ 连续, 而 $F$ 于 $f^{[-1]} [y]$ 连续, 故 $F \circ f^{[-1]}$ 于 $y$ 连续; 因为 (在 $M \cap J$ 上) $F$ 不取零值, 故 $1/(F \circ f^{[-1]})$ 亦于 $y$ 连续. 所以, $f^{[-1]}$ 于 $y$ 可导, 且导数为 $1/F[f^{[-1]} [y]]$.
\end{proof}

最后, 我给出一个十分有用的事实; 不过, 由于没有足够多的工具, 我就不论证了.

\begin{theorem}
    指数函数、馀弦函数、正弦函数于其定义域的每一点都可导. 具体地说:
    \begin{itemize}
        \item $\mathrm{exp}$: $\mathbb{R} \to (0, +\infty)$ 于 $\mathbb{R}$ 的任意一点 $x$ 都可导, 且导数为 $\mathrm{exp} [x]$.
        \item $\mathrm{cos}$: $\mathbb{R} \to [-1, 1]$ 于 $\mathbb{R}$ 的任意一点 $x$ 都可导, 且导数为 $-\mathrm{sin} [x]$,
        \item $\mathrm{sin}$: $\mathbb{R} \to [-1, 1]$ 于 $\mathbb{R}$ 的任意一点 $x$ 都可导, 且导数为 $\mathrm{cos} [x]$.
    \end{itemize}
\end{theorem}

\section{无变量的导数计算}

这是本章的重点; 这也是本书的一个重点. 不过, 不重要地, ``无变量的导数计算'' 的 ``导数'' 跟前面的 ``导数'' 不是一个词, 但仍有联系.

\begin{definition}
    设 $f$ 是区间 $I$ 上的函数. 若 $f$ 于 $I$ 的每一点都可导, 则定义函数
    \begin{align*}
        \text{$\mathrm{D}[f]$:} \quad
        I & \to K,                          \\
        x & \mapsto \text{($f$ 于 $x$ 的导数)},
    \end{align*}
    其中陪域 $K$ 可视情况而定 (除非特别说明, 我们不在意陪域). 我们称 $\mathrm{D}[f]$ 为 $f$ 的\emph{导函数} (亦可简称其为 $f$ 的\emph{导数}).
\end{definition}

抽象地, $\mathrm{D}[f]$ 的 $\mathrm{D}$ 本身就是一个函数; 不过, 这是一个变函数为函数的函数 (有点儿绕). 有时, 在不引起误会的时候, 我们也可写 $\mathrm{D}[f]$ 为 $\mathrm{D}f$; 毕竟, 至少在本书里, $\mathrm{D}$ 也只跟函数 ``作用''.

当然, 我又忘记了一件事: 若 $f$ 于 $I$ 的每一点都可导, 我们就说 $f$ 是 $I$ 上的\emph{可导函数}.

曾经, 我们写 ``$f$ 于 $x$ 的导数是 $\ell$''; 现在, 我们总算能简便地表此事以 $\mathrm{D}[f][x] = \ell$ 或 $\mathrm{D}f[x] = \ell$. 乘热打铁, 我们用简单的话转述前节的结论.

\begin{theorem}
    设 $f$, $g$ 都是区间 $I$ 上的可导函数.
    \begin{itemize}
        \item $f + g$ 也是 $I$ 上的可导函数, 且
              \begin{align*}
                  \mathrm{D} [f + g] = \mathrm{D} f + \mathrm{D} g.
              \end{align*}
        \item 设 $k$ 为常数. 则 $kf$ 也是 $I$ 上的可导函数, 且
              \begin{align*}
                  \mathrm{D} [kf] = k\mathrm{D} f.
              \end{align*}
        \item $fg$ 也是 $I$ 上的可导函数, 且
              \begin{align*}
                  \mathrm{D} [fg] = \mathrm{D} f \cdot g + f \cdot \mathrm{D} g.
              \end{align*}
        \item 设 $f$ 不取零值. 则 $g/f$ 也是 $I$ 上的可导函数, 且
              \begin{align*}
                  \mathrm{D}\, \frac{g}{f} = \frac{\mathrm{D} g \cdot f - g \cdot \mathrm{D} f}{f^2}.
              \end{align*}
    \end{itemize}
\end{theorem}

\begin{theorem}
    设 $I$, $J$ 为区间. 设 $f$: $I \to J$, $g$: $J \to K$ 都是可导函数. 则 $g \circ f$ 也是可导函数, 且
    \begin{align*}
        \mathrm{D} [g \circ f] = (\mathrm{D}g \circ f) \cdot \mathrm{D}f.
    \end{align*}
\end{theorem}

\begin{remark}
    有时, 我们也写 $g \circ f$ 为 $g[f]$; 相应地, 也可写链规则为
    \begin{align*}
        \mathrm{D} [g[f]] = \mathrm{D}g[f] \cdot \mathrm{D}f.
    \end{align*}
    我们约定 $\mathrm{D}g[f]$ 表示 $(\mathrm{D}g)[f]$; 这跟 $\mathrm{D}g \circ f$ 表示 $(\mathrm{D}g) \circ f$ 是一致的. 不过, 由于 $\mathrm{D}g[f]$ 比 $\mathrm{D}g \circ f$ 紧凑, 我们省去了一对圆括号.
\end{remark}

\begin{theorem}
    设 $I$, $J$ 为区间. 设 $f$: $I \to J$ 严单调, 且有反函数 $f^{[-1]}$: $J \to I$. 设 $f$ 是可导函数, 且 $\mathrm{D}f$ 不取零值. 则 $f^{[-1]}$ 也是可导函数, 且
    \begin{align*}
        \mathrm{D} f^{[-1]} = \frac{1}{\mathrm{D}f \circ f^{[-1]}}.
    \end{align*}
\end{theorem}

\begin{theorem}
    导数表 I:
    \begin{itemize}
        \item $\mathrm{D} c = 0$, 此处 $c$ 为常函数.
        \item $\mathrm{D}\, \iota^n = n \, \iota^{n - 1}$ ($n$ 为整数).
        \item $\mathrm{D}\, \mathrm{exp} = \mathrm{exp}$.
        \item $\mathrm{D}\, \mathrm{cos} = -\mathrm{sin}$.
        \item $\mathrm{D}\, \mathrm{sin} = \mathrm{cos}$.
    \end{itemize}
\end{theorem}

用这四个定理, 我们可以\emph{清楚地}、\emph{有条理地}计算常见的函数的导 (函) 数. 我举一些例; 您可以拿本书的计算过程跟传统的导数计算过程比较.

\begin{example}
    因为 $\mathrm{tan} = {\mathrm{sin}}\,/\,{\mathrm{cos}}$, 故
    \begin{align*}
        \mathrm{D}\, \mathrm{tan}
        = {} & \mathrm{D}\, \frac{\mathrm{sin}}{\mathrm{cos}}                                                                     \\
        = {} & \frac{\mathrm{D}\, \mathrm{sin} \cdot \mathrm{cos} - \mathrm{sin} \cdot \mathrm{D}\, \mathrm{cos}}{\mathrm{cos}^2} \\
        = {} & \frac{\mathrm{cos}^2 + \mathrm{sin}^2}{\mathrm{cos}^2}                                                             \\
        = {} & \mathrm{cos}^{-2} = 1 + \mathrm{tan}^2.
    \end{align*}
\end{example}

\begin{example}
    因为 $\mathrm{exp}$: $\mathbb{R} \to (0, +\infty)$ 严增, $\mathbb{R}$, $(0, +\infty)$ 都是区间, 且 $\mathrm{D}\, \mathrm{exp} = \mathrm{exp} > 0$, 故 $\mathrm{ln}$: $(0, +\infty) \to \mathbb{R}$ 也是可导函数, 且
    \begin{align*}
        \mathrm{D}\, \mathrm{ln} = \frac{1}{\mathrm{D}\, \mathrm{exp} \circ \mathrm{ln}} = \frac{1}{\mathrm{exp} \circ \mathrm{ln}} = \frac{1}{\iota}.
    \end{align*}

    因为 $\mathrm{sin}$: $(-2\pi/4, 2\pi/4) \to (-1, 1)$ 严增, $(-2\pi/4, 2\pi/4)$, $(-1, 1)$ 都是区间, 且 $\mathrm{D}\, \mathrm{sin} = \mathrm{cos} > 0$, 故 $\mathrm{arcsin}$: $(-1, 1) \to (-2\pi/4, 2\pi/4)$ 也是可导函数, 且
    \begin{align*}
        \mathrm{D}\, \mathrm{arcsin} = \frac{1}{\mathrm{D}\, \mathrm{sin} \circ \mathrm{arcsin}} = \frac{1}{\mathrm{cos} \circ \mathrm{arcsin}} = \frac{1}{\mathrm{sqrt} \circ (1 - \iota^2)}.
    \end{align*}

    因为 $\mathrm{tan}$: $(-2\pi/4, 2\pi/4) \to \mathbb{R}$ 严增, $(-2\pi/4, 2\pi/4)$, $\mathbb{R}$ 都是区间, 且 $\mathrm{D}\, \mathrm{tan} = 1 + \mathrm{tan}^2 > 0$, 故 $\mathrm{arctan}$: $\mathbb{R} \to (-2\pi/4, 2\pi/4)$ 也是可导函数, 且
    \begin{align*}
        \mathrm{D}\, \mathrm{arctan} = \frac{1}{\mathrm{D}\, \mathrm{tan} \circ \mathrm{arctan}} = \frac{1}{(1 + \mathrm{tan}^2) \circ \mathrm{arctan}} = \frac{1}{1 + \iota^2}.
    \end{align*}

    设 $n$ 为正整数. 因为 $\mathrm{\iota}^n$: $(0, +\infty) \to (0, +\infty)$ 严增, $(0, +\infty)$ 是区间, 且 $\mathrm{D}\, \mathrm{\iota}^n = n \cdot \iota^{n-1} > 0$, 故 $\iota^{1/n}$: $(0, +\infty) \to (0, +\infty)$ 也是可导函数, 且
    \begin{align*}
        \mathrm{D}\, \iota^{1/n} = \frac{1}{\mathrm{D}\, \mathrm{\iota}^n \circ \iota^{1/n}} = \frac{1}{(n \cdot \iota^{n-1}) \circ \iota^{1/n}} = \frac{1}{n} \iota^{1/n - 1}.
    \end{align*}
    设 $m$ 为整数. 则 $\iota^{m/n} = (\iota^{1/n})^m = \iota^m \circ \iota^{1/n}$. 故
    \begin{align*}
        \mathrm{D}\, \iota^{m/n} = ({\mathrm{D}\, \iota^m} \circ {\iota^{1/n}}) \cdot {\mathrm{D}\, \iota^{1/n}} = ((m \cdot \iota^{m-1}) \circ \iota^{1/n}) \cdot \frac{1}{n} \iota^{1/n - 1} = \frac{m}{n} \iota^{m/n - 1}.
    \end{align*}
    也就是说, 对任意有理数 $r$, $\mathrm{D}\, \iota^r = r\,\iota^{r - 1}$.
\end{example}

\begin{example}
    定义符号函数:
    \begin{align*}
        \text{$\mathrm{sign}$:} \quad
        \mathbb{R} & \to \{ 1, 0, -1 \},  \\
        t          & \mapsto \begin{cases}
                                 1,  & t > 0; \\
                                 0,  & t = 0; \\
                                 -1, & t < 0.
                             \end{cases}
    \end{align*}
    设 $I$ 是某个不含 $0$ 的区间. 故 $\mathrm{sign}$ (在 $I$ 上的限制) 是常函数, 且导数为 $0$.

    不难看出, $\mathrm{abs} = \mathrm{sign} \cdot \iota$. 所以,
    \begin{align*}
        \mathrm{D}\, \mathrm{abs} = {\mathrm{D}\, \mathrm{sign}} \cdot \iota + \mathrm{sign} \cdot {\mathrm{D}\, \iota} = \mathrm{sign} = \frac{\mathrm{abs}}{\iota}.
    \end{align*}
    由此, 我们可计算 $\mathrm{ln} \circ \mathrm{abs}$ 的导数:
    \begin{align*}
        \mathrm{D} [\mathrm{ln} \circ \mathrm{abs}] = (\mathrm{D}\, \mathrm{ln} \circ \mathrm{abs}) \cdot \mathrm{D}\, \mathrm{abs} = \bigg(\frac{1}{\iota} \circ \mathrm{abs} \bigg) \cdot \frac{\mathrm{abs}}{\iota} = \frac{1}{\iota}.
    \end{align*}
\end{example}

我们添加上面的计算结果到导数表 I, 就得到了一张较为完善的导数表 II. 以后, 我们的导数计算十分依赖此表与导数表 I 前的三个定理.

\begin{theorem}
    导数表 II:
    \begin{itemize}
        \item $\mathrm{D} c = 0$, 此处 $c$ 为常函数.
        \item $\mathrm{D}\, \iota^r = r \, \iota^{r - 1}$ ($r$ 为有理数).
        \item $\mathrm{D}\, \mathrm{exp} = \mathrm{exp}$.
        \item $\mathrm{D}\, \mathrm{cos} = -\mathrm{sin}$.
        \item $\mathrm{D}\, \mathrm{sin} = \mathrm{cos}$.
        \item $\mathrm{D}\, \mathrm{tan} = \mathrm{cos}^{-2} = 1 + \mathrm{tan}^2$.
        \item $\mathrm{D}\, \mathrm{ln} = \mathrm{D} [\mathrm{ln} \circ \mathrm{abs}] = \iota^{-1}$.
        \item $\mathrm{D}\, \mathrm{arcsin} = \mathrm{sqrt}^{-1} \circ (1 - \iota^2)$.
        \item $\mathrm{D}\, \mathrm{arctan} = 1/(1 + \iota^2)$.
        \item $\mathrm{D}\, \mathrm{abs} = \mathrm{abs}/\iota = \mathrm{sign}$.
        \item $\mathrm{D}\, \mathrm{sqrt} = \iota^{-1} \circ (2\,\mathrm{sqrt})$.
    \end{itemize}
\end{theorem}

\begin{example}
    设 $f$, $g$ 是区间 $I$ 上的可导函数, 且 $f > 0$. 求 $\mathrm{D}[f^g]$.

    事实上, $f^g$ 就是 $\mathrm{e}^{g(\mathrm{ln} \circ f)}$, 也就是 $\mathrm{exp} \circ (g(\mathrm{ln} \circ f)) = \mathrm{exp} \circ (g\, \mathrm{ln}[f])$. 所以
    \begin{align*}
        \mathrm{D}[f^g]
        = {} & \mathrm{D}[\mathrm{exp} \circ (g\, \mathrm{ln}[f])]                    \\
        = {} & ({\mathrm{D}\,\mathrm{exp}} \circ (g\, \mathrm{ln}[f]))
        \cdot \mathrm{D}[g\, \mathrm{ln}[f]]                                          \\
        = {} & (\mathrm{exp} \circ (g\, \mathrm{ln}[f]))
        \cdot (\mathrm{D}g \cdot \mathrm{ln}[f] + g \cdot \mathrm{D}[\mathrm{ln}[f]]) \\
        = {} & f^g \cdot \mathrm{D}g \cdot \mathrm{ln}[f]
        + f^g \cdot g \cdot (\mathrm{D}\,\mathrm{ln} \circ f) \cdot \mathrm{D}f       \\
        = {} & f^g \, \mathrm{ln}[f] \, \mathrm{D}g
        + g f^{g - 1}\,\mathrm{D}f.
    \end{align*}
\end{example}

\begin{example}
    设
    \begin{align*}
        \text{$f$:} \quad
        [0, +\infty) & \to \mathbb{R},                                                    \\
        x            & \to \mathrm{ln} \Bigg[ x^2 \mathrm{e}^{x} + \sqrt{1 + x^3} \Bigg].
    \end{align*}
    求 $\mathrm{D}f$.

    本问题定义 $f$ 时, 使用了带变量的记号. 为了方便地计算 $f$ 的导数, 我们不妨先无变量地表达 $f$:
    \begin{align*}
        g & = \iota^2 \cdot \mathrm{exp} + \mathrm{sqrt} \circ (1 + \iota^3), \\
        f & = \mathrm{ln} \circ g.
    \end{align*}
    从而
    \begin{align*}
        \mathrm{D}f
        = {} & \mathrm{D}[\mathrm{ln} \circ g]                                        \\
        = {} & (\mathrm{D}\,\mathrm{ln} \circ g) \cdot \mathrm{D}g                    \\
        = {} & g^{-1} \cdot \mathrm{D}[\iota^2 \cdot \mathrm{exp}]
        + g^{-1} \cdot \mathrm{D}[\mathrm{sqrt} \circ (1 + \iota^3)]                  \\
        = {} & g^{-1} \cdot (\mathrm{D}\,\iota^2 \cdot \mathrm{exp}
        + \iota^2 \cdot \mathrm{D}\,\mathrm{exp})                                     \\
             & \qquad + g^{-1} \cdot ((\mathrm{D}\,\mathrm{sqrt} \circ (1 + \iota^3))
        \cdot \mathrm{D}[1 + \iota^3])                                                \\
        = {} & g^{-1} \cdot (2\iota + \iota^2) \cdot \mathrm{exp}
        + g^{-1} \cdot \frac{1}{2\sqrt{1 + \iota^3}} \cdot 3\iota^2                   \\
        = {} & \frac{\iota\,(2 + \iota)\,\mathrm{exp}
            + \dfrac{3\iota^2}{2\sqrt{1 + \iota^3}}}
        {\iota^2 \,\mathrm{exp} + \sqrt{1 + \iota^3}}.
    \end{align*}
\end{example}
